\chapter*{UNIX UN SISTEMA OPERATIVO REVOLUCIONARIO(SU PASADO, PRESENTE Y FUTURO)}

\section*{Nacimiento de UNIX}
Hace  varios años  atrás un programador se sentó y  desarrolló   en un mes lo que se  convertiría  en una de las piezas más importantes del software jamás creado.

Todo surgió en agosto de 1969, en donde  Ken Thompson, un programador  de los laboratorios Bell de AT \& T, observó la salida de su esposa y su hijo pequeño  como una gran 
oportunidad para poner todas sus ideas en un nuevo proyecto, y era la creación de un nuevo sistema  operativo, Ken Thompson escribió la primera versión de Unix en una 
minicomputadora PDP-7 usando solamente lenguaje ensamblador, pero Ken no estuvo solo, trabajo con algunos  colegas  de su  trabajo, como  Dennis Ritchie y Doug Mcllroy  que 
le aportaron  grandes ideas y mejoras  a la primera versión  de Unix el sistema operativo  elaborado por Ken.

Unix  es  un sistema operativo  creado sobre un filosofía  de que  menos es más, esta  filosofía  se basa en que un sistema operativo no tiene que ser caro  ya sea  por el 
costo del equipo  o por el esfuerzo humano, Dennis Ritchie  dijo  en una entrevista “espero que los usuario de Unix  encuentres las características más importantes del 
sistema  que son su simplicidad, elegancia y facilidad de uso”, esta  características  muy seguramente son las  que han  llevado a Unix a ser  uno de los sistemas operativos favoritos en sus años de desarrollo.

\section*{Unix y sus Primeros Pasos}
No siempre Unix  fue  tan exitoso, en sus comienzos no tuvo  el éxito de repente, debido a que todavía no era un sistema operativo potente, en 1971 fue portado  a la 
minicomputadora PDP-11 una plataforma más potente a la minicomputadora PDP-7 que  era en donde estaba escrita originalmente, y  en  1972  Ritchie  escribió  Unix  en  
lenguaje  C, esto aumentó en una gran medida la portabilidad del sistema operativo, llegando al tan anhelado  sueño de lanzarlo al mercado de los sistemas operativos y 
competir con grandes empresas, y convirtiéndose  no solo en un sistema operativo de pocos  usuario  en  los laboratorios Bell,  si no en un sistema operativo interactivo con varios usuario , esto se logró  a la gran publicidad  y al  impacto que le generó  a la sociedad  de esa época, llevando a   ken  a un estado  en el que estaba inundado de peticiones de Unix.

\section*{Cielo de Hackers}
Thompson y Ritchie fueron los “hackers” de la época , con esa palabra se les refería a ellos, juntos combinan la creatividad poco común, la fuerza bruta de inteligencia y el aceite de la medianoche para resolver los problemas  de software que otros  no sabían que existían, el enfoque y el código que ellos escribieron atrajeron gran cantidad de programadores  provenientes de universidades y más tarde las  compañías , ellos  lograron esto  sin la necesidad de tener  un gran presupuesto como IBM u otras empresas 
gigantes en el mundo de los sistemas operativos,  esta gran atracción que tenía Unix en los programadores era debido  a que los programas de código estaban siempre  
disponibles a modificar, Thompson y Ritchie  aprovecharon esto para  mejorar su sistema operativo, ellos se basaban  en las ideas o sugerencias de los demás programadores  
que usaban Unix y reescribían el sistema operativo generando una versión actualizada que cumplia las necesidades de los usuarios.

\section*{UNIX, ¿POR QUÉ SU ÉXITO?}
El gran éxito de unix no se debe a la suerte, se debe a  que desde sus inicios tuvieron  ideales claros los cuales siguen manteniendo, por los que se destaca que fue creado 
por y para programadores,  haciendo que este fuera  interactivo y tuviera  herramientas adecuadas para el desarrollo. y junto con su capacidad multiusuario, su potente 
shell, sistema de archivos podemos sintetizar como el secreto del enorme éxito que ha caracterizado a Unix. 


 