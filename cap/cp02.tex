\chapter*{UNIX UN SISTEMA OPERATIVO REVOLUCIONARIO(SU PASADO, PRESENTE Y FUTURO)}

\section*{UNIX ¿De dónde surgió?}
En 1925, AT\&T  una compañía estadounidense de telecomunicaciones  creó una nueva unidad llamada Laboratorios Telefónicos Bell (Bell Labs) en donde hace  varios años  un             programador se sentó y  desarrolló   en un mes lo que se  convertiría  en una de las piezas más importantes del software jamás creado.

Todo surgió en agosto de 1969, donde Ken Thompson, un programador  de los laboratorios Bell, ideó  un nuevo proyecto, y era la creación de un sistema  operativo, Ken 
Thompson escribió la primera versión de Unix en una minicomputadora PDP-7 usando solamente lenguaje ensamblador, pero nunca  estuvo solo, trabajo junto a  Dennis Ritchie y 
Doug Mcllroy dos compañeros de empresa, que le aportaron  grandes ideas y mejoras  a la primera versión  de Unix.


\section*{Unix y sus Primeros Pasos}
No siempre Unix fue tan exitoso, en sus comienzos no tuvo el éxito de repente, debido a que todavía no era un sistema operativo potente, en 1971 fue portado a la 
minicomputadora PDP-11 una plataforma más potente a la minicomputadora PDP-7 que era en donde estaba escrita originalmente, y en 1972 Ritchie escribió Unix en lenguaje C, 
esto aumentó en una gran medida la portabilidad del sistema operativo, y convirtiéndolo en un sistema operativo de pocos usuarios en uno más interactivo y con varios 
usuario, llegando así al tan anhelado sueño de ser lanzado al mercado de los sistemas operativos y competir con grandes empresas, todo esto se logró en gran parte a la  
publicidad y al impacto que le generó a la sociedad de esa época, llevando a Unix como uno de los sistemas operativos más cotizados.


\section*{Gloria de Hackers}
Thompson y Ritchie fueron "hackers" en su época, con esa palabra se les refería, juntos combinaban la creatividad poco común, la fuerza de la inteligencia y el trabajo de 
la medianoche para resolver los problemas de software que otros no sabían que existían, el enfoque y el código que ellos escribieron atrajeron gran cantidad de 
programadores provenientes de universidades y más tarde las  compañías, ellos lograron esto sin la necesidad de tener un gran presupuesto como IBM u otras empresas 
gigantes en el mundo de los sistemas operativos,  esta gran atracción que tenía Unix en los programadores era debido a que los programas de código estaban siempre  
disponibles a modificar, Thompson y Ritchie aprovecharon esto para mejorar su sistema operativo, ellos se basaban en las ideas o sugerencias de los demás programadores que 
usaban Unix y reescribían el sistema operativo generando una versión actualizada que cumplía las necesidades de los usuarios.

\section*{UNIX, ¿POR QUÉ SU ÉXITO?}
El gran éxito de Unix se puede sintetizar en algunos aspectos muy importantes que lo hacen resaltar con respecto a otros sistemas.

\begin{itemize}
	\item Capacidad multitarea.
	\item Capacidad multiusuario.
	\item Portabilidad.
	\item Programas de UNIX.
	\item Biblioteca de aplicaciones.
\end{itemize} 

Multitarea esto ahorraba tiempo ya que el usuario no deba esperar a que termine una aplicación antes de iniciar otra también.
 
Multiusuario esta capacidad permite a varios usuarios acceder al mismo documento de modo que los cambios de un usuario no sobreescriban los cambios de otro usuario.                                                                  

La portabilidad permite que se pueda cambiar de marca sin tener ningún problema.

Los programas se pueden agregar o quitar dependiendo las necesidades.

Una vez Unix salió de Bell Labs muchos programadores desarrollaron aplicaciones para Unix hoy hay cientos de aplicaciones UNIX que se pueden comprar a los vendedores de 
terceras partes, además de las aplicaciones que vienen con UNIX.


\section*{NO TODO PUEDE SER BUENO}
Sin embargo como todo software presenta altibajos tanto en funcionalidades como en desarrollo, calidad o seguridad.
                                           
Como es el reciente caso de la llamada operación "Windigo" una campaña cibercriminal que fue dada al descubierto por el laboratorio de Eset junto con CERT-Bund y otros 
organismos que según estos Windigo logró tomar el control de cerca de 25.000 servidores Unix y Linux en todo el mundo infectando a más 500.000. Windigo una vez a 
comprometido al servidor los usa para tomar credenciales SSH, para redirigir a  los visitantes de los sitios web allí alojados a contenido malicioso  y para enviar spam que 
llegó a un promedio de 35 millones de mensajes diarios sin embargo no se explota alguna vulnerabilidad del sistema operativo pero esto es algo a tener muy en cuenta ya que 
se calcula que el 60\% de internet se encuentra alojado en servidores Unix y Linux.

\section*{Unix en la Actualidad}
Aunque Unix no pudo mantener el boom que genero en su momento Unix se mantiene a la vanguardia ya que se continúan lanzando productos para este sistema  como lo es PROTECT-
UX lanzado recientemente por  Computer Security Products Protect-UX es una solución que mejora considerablemente la seguridad y reduce la carga administrativa en los 
sistemas Linux y Unix también ofreciendo soluciones a problemas de archivos este tipo de productos son enfocados en la actualidad y futuro de Unix el cual apunta es a ser  
usado en servidores web de gran tráfico.

\section*{EL LEGADO DE UNIX}
Unix  a dejado una marca en lo referente a los sistemas operativos ya que ha dejado  muchos sistemas operativos basados en este los cuales son bastante conocidos y con 
miles de usuarios y desarrollados varios campos del software y por diferentes compañías con distintos fines  podemos ver algunos ejemplos como lo son Linux, iOS, Solaris, 
BSD,HP OSS, que fueron basados en Unix


 