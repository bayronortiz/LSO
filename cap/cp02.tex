\chapter*{UNIX El sistema operativo revolucionario}

\section*{UNIX ¿De dónde surgió?}
A fines de 1960, el Instituto Tecnológico de Massachusetts, unos  de los más prestigiosos actualmente, los Laboratorios Bell de AT\&T y General Electric trabajaban en un sistema 
operativo experimental llamado Multics (Multiplexed Information and Computing System), desarrollado para ejecutarse en una computadora central (mainframe) modelo GE-645. El 
objetivo del proyecto era desarrollar un gran sistema operativo interactivo que contase con muchas innovaciones, entre ellas mejoras en las políticas de seguridad.
 
El proyecto consiguió generar versiones para producción, pero las primeras versiones contaban con un pobre rendimiento. Los laboratorios Bell de AT\&T decidieron desvincularse y 
dedicar sus recursos a otros proyectos. Uno de los programadores de los laboratorios Bell, Ken Thompson, siguió trabajando para la computadora GE-6354 y escribió un juego llamado 
SpaceTravel, (Viaje espacial). Sin embargo, descubrió que el juego era lento en la máquina de General Electric y resultaba realmente caro, algo así como 75 dólares de EE.UU. por 
cada partida. De este modo, Thompson escribió nuevamente el programa, con ayuda de Dennis Ritchie, en lenguaje ensamblador, para que se ejecutase en una computadora DEC PDP-7. 
Esta experiencia, junto al trabajo que desarrolló para el proyecto Multics, condujo a Thompson a iniciar la creación de un nuevo sistema operativo para la DEC PDP-7, Thompson y 
Ritchie lideraron un grupo de programadores, entre ellos a RuddCanaday, en los laboratorios Bell, para desarrollar tanto el sistema de ficheros como el sistema operativo 
multitarea . A lo anterior, agregaron un intérprete de órdenes (o intérprete de comandos) y un pequeño conjunto de programas.
 
El proyecto fue bautizado UNICS, como acrónimo Uniplexed Information and Computing System, pues solo prestaba servicios a dos usuarios (que de acuerdo con Andrew Tanenbaum, era 
solo a un usuario). No se conoce la razón pero el nombre fue cambiado a UNIX, dando origen al legado que continúa hasta nuestros días. Hasta ese instante, no había existido  
apoyo económico por parte de los laboratorios Bell, pero eso cambió cuando el Grupo de Investigación en Ciencias de la Computación decidió utilizar UNIX en una máquina superior a 
la PDP-7. Thompson y Ritchie lograron cumplir con la solicitud de agregar herramientas que permitieran el procesamiento de textos a UNIX en una máquina PDP-11/20, y como 
consecuencia de ello consiguieron el apoyo económico de los laboratorios Bell.
 
Fue así como por vez primera, en 1970, se habla oficialmente del sistema operativo UNIX  ejecutado en una PDP-11/20. Se incluía en él un programa para dar formato a textos llanos 
(runoff) y un editor de texto. Tanto el sistema operativo como los programas fueron escritos en el lenguaje ensamblador de la PDP-11/20. Este "sistema de procesamiento de texto" 
inicial, compuesto tanto por el sistema operativo como de runoff y el editor de texto, fue utilizado en los laboratorios Bell para procesar las solicitudes de patentes que ellos 
recibían. Pronto, runoff el editor de texto de Unix evolucionó hasta convertirse en troff, el primer programa de edición electrónica que permitía realizar composición 
tipográfica. El 3 de noviembre de 1971 Thomson y Ritchie publicaron un manual de programación de UNIX (título original en inglés: "UNIX Programmer's Manual"). En 1972 se tomó la 
decisión de escribir nuevamente UNIX, pero esta vez en el lenguaje de programación C.

Este cambio significaba que UNIX podría ser fácilmente modificado para funcionar en otras computadoras (de esta manera, se volvía portable) y así otras variaciones podían ser 
desarrolladas por otros programadores. Ahora, el código era más conciso y compacto, lo que se tradujo en un aumento en la velocidad de desarrollo de UNIX. AT\&T puso a UNIX a 
disposición de universidades y compañías, también al gobierno de los Estados Unidos, a través de licencias.
 
Una de estas licencias fue otorgada al Departamento de Computación de la Universidad de California, con sede en Berkeley. En 1975 esta institución desarrolló y publicó su propio 
sucedáneo de UNIX, conocida como Berkeley Software Distribution (BSD), que se convirtió en una fuerte competencia para la familia UNIX de AT\&T. Mientras tanto, AT\&T creó una 
división comercial denominada Unix Systems Laboratories para la explotación comercial del sistema operativo. El desarrollo prosiguió, con la entrega de las versiones 4, 5 y 6 en 
el transcurso de 1975. Estas versiones incluían los pipes o tuberías, lo que permitió dar al desarrollo una orientación modular respecto a la base del código, consiguiendo 
aumentar aún más la velocidad de desarrollo.
 
Ya en 1978, cerca de 600 o más máquinas estaban ejecutándose con alguna de las distintas encarnaciones de UNIX. La versión 7, la última versión del UNIX original con amplia 
distribución, entró en circulación en 1979. Las versiones 8, 9 y 10 se desarrollaron durante la década de 1980, pero su circulación se limitó a unas cuantas universidades, a 
pesar de que se publicaron los informes que describían el nuevo trabajo. Los resultados de esta investigación sirvieron de base para la creación de Plan 9 from Bell Labs, un 
nuevo sistema operativo portable y distribuido, diseñado para ser el sucesor de UNIX en investigación por los Laboratorios Bell. AT\&T entonces inició el desarrollo de UNIX 
System III, basado en la versión 7, como una variante  comercial y así vendía el producto de manera directa.
 
La primera versión se lanzó en 1981. A pesar de lo anterior, la empresa subsidiaria Western Electric seguía vendiendo versiones antiguas de Unix basadas en las distintas 
versiones hasta la séptima .AT\&T decidió combinar varias versiones desarrolladas en distintas universidades y empresas, dando origen en 1983 al Unix System V Release 1. Esta 
versión presentó características tales como el editor Vi y la biblioteca curses, desarrolladas por Berkeley Software Distribution en la Universidad de California, Berkeley. 
También contaba con compatibilidad con las máquinas VAX de la compañía DEC. Hacia 1991, un estudiante de ciencias de la computación de la Universidad de Helsinki, llamado Linus 
Torvalds desarrolló un núcleo para computadoras con arquitectura x86 de Intel que emulaba muchas de las funcionalidades de UNIX y lo lanzó en forma de código abierto en 1991, 
bajo el nombre de Linux.
 
En 1992, el Proyecto GNU comenzó a utilizar el núcleo Linux junto a sus programas. En 1993, la compañía Novell adquirió la división Unix Systems Laboratories de AT\&T junto con 
su propiedad intelectual. Esto ocurrió en un momento delicado en el que Unix Systems Laboratories disputaba una demanda en los tribunales contra BSD por infracción de los 
derechos de copyright, revelación de secretos y violación de marca de mercado.
 
Aunque BSD ganó el juicio, Novell descubrió que gran parte del código de BSD fue copiada ilegalmente en UNIX System V. En realidad, la propiedad intelectual de Novell se reducía 
a unos cuantos archivos fuente. La correspondiente contra-demanda acabó en un acuerdo extrajudicial cuyos términos permanecen bajo secreto a petición de Novell. En 1995, Novell 
vendió su división UNIX comercial (es decir, la antigua Unix Systems Laboratories) a Santa Cruz Operation (SCO) reservándose, aparentemente, algunos derechos de propiedad 
intelectual sobre el software.SCO continúa la comercialización de System V en su producto UnixWare, que durante cierto tiempo pasó a denominarse Open Unix, aunque ha retomado de 
nuevo el nombre de UnixWare.


\section*{Los "Hackers"}
Thompson y Ritchie fueron "hackers" en su época, con esa palabra se les refería, juntos combinaban la creatividad poco común, la fuerza de la inteligencia y el trabajo de 
la medianoche para resolver los problemas de software que otros no sabían que existían, el enfoque y el código que ellos escribieron atrajeron gran cantidad de 
programadores provenientes de universidades y más tarde las  compañías, ellos lograron esto sin la necesidad de tener un gran presupuesto como IBM u otras empresas 
gigantes en el mundo de los sistemas operativos,  esta gran atracción que tenía Unix en los programadores era debido a que los programas de código estaban siempre  
disponibles a modificar, Thompson y Ritchie aprovecharon esto para mejorar su sistema operativo, ellos se basaban en las ideas o sugerencias de los demás programadores que 
usaban Unix y reescribían el sistema operativo generando una versión actualizada que cumplía las necesidades de los usuarios.


\section*{UNIX, ¿POR QUÉ SU ÉXITO?}
El gran éxito de Unix se puede sintetizar en algunos aspectos muy importantes que lo hacen resaltar con respecto a otros sistemas.

\begin{itemize}
	\item Capacidad multitarea.
	\item Capacidad multiusuario.
	\item Portabilidad.
	\item Programas de UNIX.
	\item Biblioteca de aplicaciones.
\end{itemize} 

Multitarea esto ahorraba tiempo ya que el usuario no deba esperar a que termine una aplicación antes de iniciar otra también.
 
Multiusuario esta capacidad permite a varios usuarios acceder al mismo documento de modo que los cambios de un usuario no sobreescriban los cambios de otro usuario.                                                                  

La portabilidad permite que se pueda cambiar de marca sin tener ningún problema.

Los programas se pueden agregar o quitar dependiendo las necesidades.

UNIX es un Sistema Operativo de propósito general con las siguientes características:

\begin{itemize}
	\item Sistema Multiusuario.
	\item Sistema Multitarea.
	\item Trabaja en tiempo compartido.
	\item Sistema interactivo.
	\item Estandarizado.
	\item Potente.
	\item Versátil.
	\item Software y sistema operativo portable.
	\item Permite ejecutar procesos en Background y Foreground.
	\item Presenta niveles altos de seguridad.
	\item Presenta una estructura jerárquica de archivos.
	\item Posee un gran número de utilidades: Editores de línea, de pantalla, formateadores, depuradores de programas entre otros.
	\item Posibilidad de comunicación UNIX – UNIX (UUCP).
	\item Fácil integración en sistemas de red.
	\item Posibilidad de usar distintos lenguajes de programación: C,Fortran, Cobol, Pascal etc.
	\item Gran cantidad de software disponible en el mercado.
	\item Manejo y control de los dispositivos físicos.
	\item Control de interrupciones.
	\item Control de procesos y tiempos.
	\item Control de archivos.
	\item Control de memoria.
	\item Características semejantes a un lenguaje de programación de alto nivel.
	\item Shell scripts.

\end{itemize}


\section*{NO TODO PUEDE SER BUENO}
Sin embargo como todo software presenta altibajos tanto en funcionalidades como en desarrollo, calidad o seguridad.
                                           
Como es el reciente caso de la llamada operación "Windigo" una campaña cibercriminal que fue dada al descubierto por el laboratorio de Eset junto con CERT-Bund y otros 
organismos que según estos Windigo logró tomar el control de cerca de 25.000 servidores Unix y Linux en todo el mundo infectando a más 500.000. Windigo una vez a 
comprometido al servidor los usa para tomar credenciales SSH, para redirigir a  los visitantes de los sitios web allí alojados a contenido malicioso  y para enviar spam que 
llegó a un promedio de 35 millones de mensajes diarios sin embargo no se explota alguna vulnerabilidad del sistema operativo pero esto es algo a tener muy en cuenta ya que 
se calcula que el 60\% de internet se encuentra alojado en servidores Unix y Linux.

\section*{Unix en la Actualidad}
Aunque Unix no pudo mantener el boom que genero en su momento Unix se mantiene a la vanguardia ya que se continúan lanzando productos para este sistema  como lo es PROTECT-
UX lanzado recientemente por  Computer Security Products Protect-UX es una solución que mejora considerablemente la seguridad y reduce la carga administrativa en los 
sistemas Linux y Unix también ofreciendo soluciones a problemas de archivos este tipo de productos son enfocados en la actualidad y futuro de Unix el cual apunta es a ser  
usado en servidores web de gran tráfico.

\section*{EL LEGADO DE UNIX}
Unix  a dejado una marca en lo referente a los sistemas operativos ya que ha dejado  muchos sistemas operativos basados en este los cuales son bastante conocidos y con 
miles de usuarios y desarrollados varios campos del software y por diferentes compañías con distintos fines  podemos ver algunos ejemplos como lo son Linux, iOS, Solaris, 
BSD,HP OSS, que fueron basados en Unix


 