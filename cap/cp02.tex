\chapter*{UNIX EL SISTEMA OPERATIVO REVOLUCIONARIO}

\section*{UNIX ¿DE DÓNDE SURGIÓ?}
A fines de 1960, el Instituto Tecnológico de Massachusetts, unos  de los más prestigiosos actualmente, los Laboratorios Bell de AT\&T y General Electric trabajaban en un sistema 
operativo experimental llamado Multics (Multiplexed Information and Computing System), desarrollado para ejecutarse en una computadora central (mainframe) modelo GE-645. El 
objetivo del proyecto era desarrollar un gran sistema operativo interactivo que contase con muchas innovaciones, entre ellas mejoras en las políticas de seguridad.
 
El proyecto consiguió generar versiones para producción, pero las primeras versiones contaban con un pobre rendimiento. Los laboratorios Bell de AT\&T decidieron desvincularse y 
dedicar sus recursos a otros proyectos. Uno de los programadores de los laboratorios Bell, Ken Thompson, siguió trabajando para la computadora GE-6354 y escribió un juego llamado 
SpaceTravel, (Viaje espacial). Sin embargo, descubrió que el juego era lento en la máquina de General Electric y resultaba realmente caro, algo así como 75 dólares de EE.UU. por 
cada partida. De este modo, Thompson escribió nuevamente el programa, con ayuda de Dennis Ritchie, en lenguaje ensamblador, para que se ejecutase en una computadora DEC PDP-7. 
Esta experiencia, junto al trabajo que desarrolló para el proyecto Multics, condujo a Thompson a iniciar la creación de un nuevo sistema operativo para la DEC PDP-7, Thompson y 
Ritchie lideraron un grupo de programadores, entre ellos a RuddCanaday, en los laboratorios Bell, para desarrollar tanto el sistema de ficheros como el sistema operativo 
multitarea . A lo anterior, agregaron un intérprete de órdenes (o intérprete de comandos) y un pequeño conjunto de programas.
 
El proyecto fue bautizado UNICS, como acrónimo Uniplexed Information and Computing System, pues solo prestaba servicios a dos usuarios (que de acuerdo con Andrew Tanenbaum, era 
solo a un usuario). No se conoce la razón pero el nombre fue cambiado a UNIX, dando origen al legado que continúa hasta nuestros días. Hasta ese instante, no había existido  
apoyo económico por parte de los laboratorios Bell, pero eso cambió cuando el Grupo de Investigación en Ciencias de la Computación decidió utilizar UNIX en una máquina superior a 
la PDP-7. Thompson y Ritchie lograron cumplir con la solicitud de agregar herramientas que permitieran el procesamiento de textos a UNIX en una máquina PDP-11/20, y como 
consecuencia de ello consiguieron el apoyo económico de los laboratorios Bell.
 
Fue así como por vez primera, en 1970, se habla oficialmente del sistema operativo UNIX  ejecutado en una PDP-11/20. Se incluía en él un programa para dar formato a textos llanos 
(runoff) y un editor de texto. Tanto el sistema operativo como los programas fueron escritos en el lenguaje ensamblador de la PDP-11/20. Este "sistema de procesamiento de texto" 
inicial, compuesto tanto por el sistema operativo como de runoff y el editor de texto, fue utilizado en los laboratorios Bell para procesar las solicitudes de patentes que ellos 
recibían. Pronto, runoff el editor de texto de Unix evolucionó hasta convertirse en troff, el primer programa de edición electrónica que permitía realizar composición 
tipográfica. El 3 de noviembre de 1971 Thomson y Ritchie publicaron un manual de programación de UNIX (título original en inglés: "UNIX Programmer's Manual"). En 1972 se tomó la 
decisión de escribir nuevamente UNIX, pero esta vez en el lenguaje de programación C.

Este cambio significaba que UNIX podría ser fácilmente modificado para funcionar en otras computadoras (de esta manera, se volvía portable) y así otras variaciones podían ser 
desarrolladas por otros programadores. Ahora, el código era más conciso y compacto, lo que se tradujo en un aumento en la velocidad de desarrollo de UNIX. AT\&T puso a UNIX a 
disposición de universidades y compañías, también al gobierno de los Estados Unidos, a través de licencias.
 
Una de estas licencias fue otorgada al Departamento de Computación de la Universidad de California, con sede en Berkeley. En 1975 esta institución desarrolló y publicó su propio 
sucedáneo de UNIX, conocida como Berkeley Software Distribution (BSD), que se convirtió en una fuerte competencia para la familia UNIX de AT\&T. Mientras tanto, AT\&T creó una 
división comercial denominada Unix Systems Laboratories para la explotación comercial del sistema operativo. El desarrollo prosiguió, con la entrega de las versiones 4, 5 y 6 en 
el transcurso de 1975. Estas versiones incluían los pipes o tuberías, lo que permitió dar al desarrollo una orientación modular respecto a la base del código, consiguiendo 
aumentar aún más la velocidad de desarrollo.
 
Ya en 1978, cerca de 600 o más máquinas estaban ejecutándose con alguna de las distintas encarnaciones de UNIX. La versión 7, la última versión del UNIX original con amplia 
distribución, entró en circulación en 1979. Las versiones 8, 9 y 10 se desarrollaron durante la década de 1980, pero su circulación se limitó a unas cuantas universidades, a 
pesar de que se publicaron los informes que describían el nuevo trabajo. Los resultados de esta investigación sirvieron de base para la creación de Plan 9 from Bell Labs, un 
nuevo sistema operativo portable y distribuido, diseñado para ser el sucesor de UNIX en investigación por los Laboratorios Bell. AT\&T entonces inició el desarrollo de UNIX 
System III, basado en la versión 7, como una variante  comercial y así vendía el producto de manera directa.
 
La primera versión se lanzó en 1981. A pesar de lo anterior, la empresa subsidiaria Western Electric seguía vendiendo versiones antiguas de Unix basadas en las distintas 
versiones hasta la séptima .AT\&T decidió combinar varias versiones desarrolladas en distintas universidades y empresas, dando origen en 1983 al Unix System V Release 1. Esta 
versión presentó características tales como el editor Vi y la biblioteca curses, desarrolladas por Berkeley Software Distribution en la Universidad de California, Berkeley. 
También contaba con compatibilidad con las máquinas VAX de la compañía DEC. Hacia 1991, un estudiante de ciencias de la computación de la Universidad de Helsinki, llamado Linus 
Torvalds desarrolló un núcleo para computadoras con arquitectura x86 de Intel que emulaba muchas de las funcionalidades de UNIX y lo lanzó en forma de código abierto en 1991, 
bajo el nombre de Linux.
 
En 1992, el Proyecto GNU comenzó a utilizar el núcleo Linux junto a sus programas. En 1993, la compañía Novell adquirió la división Unix Systems Laboratories de AT\&T junto con 
su propiedad intelectual. Esto ocurrió en un momento delicado en el que Unix Systems Laboratories disputaba una demanda en los tribunales contra BSD por infracción de los 
derechos de copyright, revelación de secretos y violación de marca de mercado.
 
Aunque BSD ganó el juicio, Novell descubrió que gran parte del código de BSD fue copiada ilegalmente en UNIX System V. En realidad, la propiedad intelectual de Novell se reducía 
a unos cuantos archivos fuente. La correspondiente contra-demanda acabó en un acuerdo extrajudicial cuyos términos permanecen bajo secreto a petición de Novell. En 1995, Novell 
vendió su división UNIX comercial (es decir, la antigua Unix Systems Laboratories) a Santa Cruz Operation (SCO) reservándose, aparentemente, algunos derechos de propiedad 
intelectual sobre el software.SCO continúa la comercialización de System V en su producto UnixWare, que durante cierto tiempo pasó a denominarse Open Unix, aunque ha retomado de 
nuevo el nombre de UnixWare.


\section*{LOS "HACKERS"}
Thompson y Ritchie fueron "hackers" en su época, con esa palabra se les refería, juntos combinaban la creatividad poco común, la fuerza de la inteligencia y el trabajo de 
la medianoche para resolver los problemas de software que otros no sabían que existían, el enfoque y el código que ellos escribieron atrajeron gran cantidad de 
programadores provenientes de universidades y más tarde las  compañías, ellos lograron esto sin la necesidad de tener un gran presupuesto como IBM u otras empresas 
gigantes en el mundo de los sistemas operativos,  esta gran atracción que tenía Unix en los programadores era debido a que los programas de código estaban siempre  
disponibles a modificar, Thompson y Ritchie aprovecharon esto para mejorar su sistema operativo, ellos se basaban en las ideas o sugerencias de los demás programadores que 
usaban Unix y reescribían el sistema operativo generando una versión actualizada que cumplía las necesidades de los usuarios.


\section*{UNIX, ¿POR QUÉ SU ÉXITO?}
El gran éxito de Unix se puede sintetizar en algunos aspectos muy importantes que lo hacen resaltar con respecto a otros sistemas.

\begin{itemize}
	\item Capacidad multitarea.
	\item Capacidad multiusuario.
	\item Portabilidad.
	\item Programas de UNIX.
	\item Biblioteca de aplicaciones.
\end{itemize} 

Multitarea esto ahorraba tiempo ya que el usuario no deba esperar a que termine una aplicación antes de iniciar otra también.
 
Multiusuario esta capacidad permite a varios usuarios acceder al mismo documento de modo que los cambios de un usuario no sobreescriban los cambios de otro usuario.                                                                  

La portabilidad permite que se pueda cambiar de marca sin tener ningún problema.

Los programas se pueden agregar o quitar dependiendo las necesidades.

UNIX es un Sistema Operativo de propósito general con las siguientes características:

\begin{itemize}
	\item Sistema Multiusuario.
	\item Sistema Multitarea.
	\item Trabaja en tiempo compartido.
	\item Sistema interactivo.
	\item Estandarizado.
	\item Potente.
	\item Versátil.
	\item Software y sistema operativo portable.
	\item Permite ejecutar procesos en Background y Foreground.
	\item Presenta niveles altos de seguridad.
	\item Presenta una estructura jerárquica de archivos.
	\item Posee un gran número de utilidades: Editores de línea, de pantalla, formateadores, depuradores de programas entre otros.
	\item Posibilidad de comunicación UNIX – UNIX (UUCP).
	\item Fácil integración en sistemas de red.
	\item Posibilidad de usar distintos lenguajes de programación: C,Fortran, Cobol, Pascal etc.
	\item Gran cantidad de software disponible en el mercado.
	\item Manejo y control de los dispositivos físicos.
	\item Control de interrupciones.
	\item Control de procesos y tiempos.
	\item Control de archivos.
	\item Control de memoria.
	\item Características semejantes a un lenguaje de programación de alto nivel.
	\item Shell scripts.
\end{itemize}

Además permite configurar diferentes entornos de trabajo. El sistema operativo UNIX tiene dos componentes fundamentales:
\begin{description}
	\item[El Kernel:] Es aquel que constituye el núcleo del sistema operativo. Actúa como interfase con el hardware del sistema, este también se encarga de realizar las funciones 
							de más bajo nivel. El (90\%) está escrito fundamentalmente en lenguaje C el resto en lenguaje ensamblador.
	\item[El Shell:] Es la interfaz del sistema operativo con sus usuarios. Este actúa como intérprete de comandos, tiene características semejantes a un lenguaje de programación 
						  de alto nivel.
\end{description}  

\section*{NO TODO PUEDE SER BUENO}
Sin embargo como todo software presenta altibajos tanto en funcionalidades como en desarrollo, calidad o seguridad.
                                           
Como se dice en un reciente estudio desarrollado por el Laboratorio de Investigación de ESET, junto con CERT-Bund (Swedish National Infrastructure for   Computing) y otros 
organismos anunciaron que se está desarrollando una campaña cibernética criminal denominada Windigo, que consiste en  la inyección de software malicioso, a lo cual se comprometen 
miles de servidores de Linux y Unix. Una vez infectado el servidor, este software malicioso roba las credenciales de las víctimas, no  solo roba las credenciales también redirige  
el tráfico de Internet hacia contenidos maliciosos o el envío de millones de mensajes de spam al día.

La compañía ESET de seguridad informática dijo que “Windigo, mientras está   pasando desapercibido por una gran  parte de los investigadores en seguridad, este lleva operando más 
de dos años y medio”. Pierre-Marc Bureau, director de los programas de seguridad de ESET, dijo que “Windigo tiene actualmente unos 10.000 servidores bajo su control”. También 
afirmó que “Este número ya es significativo si se tiene en cuenta que cada uno de estos sistemas tiene acceso a un ancho de banda considerable, una gran capacidad de 
almacenamiento, potencia de cálculo y memoria”. El equipo de investigación de seguridad de ESET ha trabajado en colaboración con la Infraestructura Nacional Sueca de Computación 
y otros organismos, señalando que una vez infectados los sistemas de las víctimas, se utilizan para redirigir el tráfico de Internet  hacia contenidos maliciosos y el envío de 
Spam.

Como Linux  y Unix  tienen miles de servidores, es casi evidente que están en un peligro inminente, debido a que Windigo es vista como una operación a gran escala, por lo que el 
daño que haría sería devastador.

Windigo usa una puerta trasera de OpenSSH, e inserta un programa que cambia la dirección web y envía correo no deseado. Varios servidores ubicados en Estados Unidos, Alemania, 
Francia y el Reino Unido se encuentran entre los equipos infectados por este  software  malicioso.


\section*{¿QUÉ  ESTÁ PASANDO  CON UNIX?}
Es  una de las tantas preguntas que   se hacen  los amantes de este sistema  operativo, como es posible  que  un sistema operativo  que  tuvo una gran influencia  en lo que  son 
hoy en día los sistemas  operativos,  día a día  se  venga  abajo , qué  futuro le  espera  a unix, "¿es  un futuro incierto? el que le espera". 	
es  una gran verdad que  mientras  Linux crece y se  afianza  en  el entorno de servidores corporativos, y  Windows   el sistema operativo con más  usuarios  refuerza  su  
posicionamiento como plataforma para aplicaciones  críticas, Unix  se  va quedando  relegado segun lo  afirma la  IDC (consultoras IDC de España).                                       

Cabe  aclarar que la mala situación por la que pasa Unix, se  debe en gran parte  por la  crisis  del punto com (.com), y  al crecimiento que  ha tenido   Linux  en el ámbito de 
los servidores de entrada, "en el  año 2001  fue una temporada  critica para  Unix"  segun lo afirma  la consultora IDC,  en ese  año bajaron radicalmente las  ventas  e  
ingresos   para  Unix, y  no obstante el crecimiento de  Linux  en el ámbito corporativo   que  cada día tenia mas  y mas acogida en el mercado, generando  una diferencia  cada 
vez menor  entre Unix  y Linux, como una lo afirmo el director de análisis de IDC en España Jaime gracia "Linux le está  robando mercado a Unix. Al menos en la parte de 
infraestructura de Internet. Además, las empresas cada vez más, perciben a Linux como un sistema muy robusto y con gran estabilidad".  	

Como la tecnología  avanza cada día mas y mas, las necesidades surgen por montones, los sistemas operativos necesitan estar a la vanguardia, esto se ha vuelto en una presión  muy 
grande para  Unix,Linuxy  Windows, que son los sistemas más utilizados  en la  actualidad.

La presión a la  que están constantemente Unix, Linux y Windows  pueden perjudicarlos o favorecerles en su desarrollo como  sistema operativo, como lo señala  un  estudio 
realizado por la consultora IDC, el cual analiza. 


\section*{LA LUCHA POR LOS DERECHOS DE UNIX}
The Open Group es un consorcio de la industria del software que brinda estándares abiertos y neutrales para el ámbito de la informática y que posee miembros tan importantes como 
Fujitsu, Hitachi, HP, IBM, Departamento de Defensa de Estados Unidos, NASA entre otros   tuvo  una disputa legal con  Novell donde en 2009 donde perdió la propiedad de la Marca 
Unix con Novell una multinacional estadounidense de software  quien es propietaria de lo que fueron los Bell Labs pero esto no acabaría allí  ya que en 2004 The scogroup una 
compañía de americana de software conocida por brindar los servicios de  scoopenserver estableció una demanda reclamando la propiedad del código fuente de Unix, incluyendo 
porciones de Linux. Luego de una larga pelea judicial se concluye que Novell es el propietario de los derechos de autor de Unix y que SCO  ha incumplido los acuerdos de 
transferencia de activos por tanto SCO debía a Novell \$2.5 millones en regalías no pagadas cosa que llevó a esta compañía a la quiebra.

Esto nos puede hacer una clara idea de lo influyente que es y de que el impacto de este sistema operativo que llevó a gigantes organizaciones a estar en grandes  disputas 
jurídicas por poseer los derechos de esta. 
 
\subsection*{LA BATALLA  ENTRE  MICROSOFT Y LINUX CONTINUA}
Microsoft sigue  intentando  en superar  a linux  de diferentes  maneras, y ha llegado  a  tal punto  de  comprar la  licencia  de  Unix,  que  es  propiedad  de  la  compañía 
SCO y  rival  de  su propio sistema  Windows, El movimiento tiene varias explicaciones. Una de ellas es práctica. Las grandes máquinas o servidores de muchas grandes empresas 
funcionan, actualmente, con Unix, pero la mayor parte de ellas están migrando al sistema Linux, de código abierto ,es decir, que se puede estudiar y modificar libremente ,y 
además es  más barato que Unix,  como Linux en su  código tiene una estructura muy parecida al de Unix, facilita el cambio  de  estas  empresas  a un sistema  libre  como lo  es  
Linux.

El interés  que tiene Microsoft  en comprar la licencia  de  Unix es debido a que  con el código de Unix en la mano, Microsoft pretende mejorar la compatibilidad de ambos 
sistemas y convencer a los clientes de Unix de que migren a Windows en lugar de a Linux.
Hay una segunda razón legal. SCO asegura que el corazón de Linux es suyo, ya que se basa en el código de Unix. Por eso, ha demandado a IBM en los tribunales estadounidenses, ya 
que asegura,que  el desarrollo de Linux que ha hecho IBM viola su propiedad intelectual, al estar basado en Unix. El sector espera que, de prosperar, a esta demanda le sigan 
otras contra el resto de las empresas que desarrollan Linux  como  HP .

Se  dice  que  con  la compra del código, Microsoft trata de evitar problemas en los tribunales y, según sus competidores, alimentar la batalla legal contra Linux.


\section*{EL LEGADO DE UNIX}
Hoy en día Unix es el nombre genérico que se le da a sistemas multiusuario y multitarea para esto existe Single Unix Specification (Especificación Única de Unix publicada por The 
Open Group)   ya que Unix posee diferentes tipos de versiones e implementaciones tanto comerciales como libres Single Unix Specification es un conjunto de estándares que  
establecen los requisitos para que un sistema pueda proclamarse Unix entre los certificados se encuentran:
\begin{description}
	\item[Las versiones comerciales como:]
\end{description}

\begin{description}
	\item[AIX:]
		Advanced Interactive eXecutive propiedad de IBM. AIX corre en los servidores IBM eServersp Series, utilizando procesadores de la familia IBM POWER de 32 y 64 bits.
		AIX cuenta con Object Data Manager (una base de datos de información del sistema). La integración de AIX del "Logical Volume Management" (administrador de volumen lógico). 
		su ultima version es AIX 7.1, lanzada en Marzo de 2012.
	
	\item[HP-UX:]
		Desarrollada y mantenida por Hewlett-Packard desde 1983 posee un entorno de trabajo flexible, potente y estable, que una gran variedad de aplicaciones en las que se 	
		encuentran simples editores de texto hasta complejos programas de diseño gráfico o cálculo científico, pasando por sistemas de control industrial que incluyen 
		planificaciones de tiempo real.
		
		Actualmente se enfoca en la seguridad como lo es el sistema de detección de intrusos IDS/9000 para HP-UX 11.x corriendo sobre máquinas HP-9000.
		
	\item[Solaris:]
		En 2009 Oracle compró Sun desde entonces este sistema se a especializado en soportar la tecnología ORACLE RAC (Real ApplicationCluster)  que permite la ejecución de una 		
		aplicación distribuida en varios servidores últimamente se incorporó la tecnología  ZFS(Zetabyte File System) un sistema de archivos con almacenamiento periódico. 		
		la ultima version es Solaris 11 publicado el 9 de noviembre de 2011.
		
	\item[Mac OS X:]
		Se trata de un UNIX completo, aprobado por The Open Group(Conocidos por publicar la definición de Unix ). Su gran diferencia es que posee una interfaz gráfica propia  
		llamada Aqua, principalmente desarrollada en Objective-C y no en c. este sistema a sido  comercializado y vendido por Apple Inc.Ha sido incluido en su gama de computadoras 
		Macintosh desde el año de 2002.
\end{description}

\begin{description}
	\item[Las versiones no comerciales como:]
\end{description}

\begin{description}
	\item[FreeBSD,OpenBSD,NetBSD:]
		Herederos del antiguo BSD 4.4  los tres bastantes similares pero cada uno especializado en algo en especial como seguridad multiplataforma y fácil uso.
	
	\item[Linux:]
		Linux y Unix  se  suelen diferenciar en textos y cuando se habla de ellos pero en términos de habilidades Linux solo es una variante suplementaria de Unix el cual cuenta 
		con diferentes tipos de distribuciones muy conocidas como Debian, Ubuntu, RedHat etc.  Solo con este este ejemplo podemos ver como este software a inspirado incluso 
		software libre.
\end{description}


\section*{UNIX EN EL SÉPTIMO ARTE}
Unix ha hecho apariciones en diferentes películas bastante conocidas y aunque no son apariciones directas son pequeñas escenas en las que se hace referencia a este sistema 
operativo por lo que da a entender de que no es ningún product placement si no que han sido por el impacto que este sistema ha generado entre los casos con más impacto curiosos e 
incluso graciosos encontramos:

\subsection*{JURASSIC PARK}
En esta famosa película de Steven el sistema de seguridad del parque se manejado con  un Macintosh Quadra 700 pero se usa una interfaz con la que se concluye que se usa Unix.

\subsection*{BATMAN}
Un famoso sitio web analizo también la 'Baticomputadora'  de las películas de los años 80 y 90.que si se tratara de un computador central es muy posible fuese un clon de UNIX 
personalizado, explica que Batman utilizó la máquina T932 en 1995, basada en UNIX.

\subsection*{CHEQUE EN BLANCO}
El protagonista se hace pasar por asesor de un jefe multimillonario inventado al cual le pone el nombre de Macintosh. El nombre lo saca de su Macintosh Performa 600 pero le añade 
una K para que no lo relacionen con Apple.

\subsection*{DR. WHO}
En esta reconocida serie la mayoría de computadores corren con Ubuntu y se puede identificar por la pantalla de arranque.

\subsection*{ANTITRUST}
En esta película también podemos ver:
\begin{itemize}
	\item A Scott McNeeley, CEO de Sun Microsystems de la época, también entregando un premio.
	\item En una imagen en blanco y negro, aparece claramente un hacker usando un sombrero rojo tipo Fedora, en alusión a Red Hat.
	\item Parte del código de bzip2, cuando muestran un "interesante código de compresión".
	\item También se puede ver algo de código HTML, que fue tomado desde el mismo código desde el sitio Internet Movie Data Base (imdb).
	\item Uso de comandos similares a Unix, en donde uno de ellos es "show -p 1984", en donde el numero del proceso es una alusión al libro de George Orwell: 1984.
\end{itemize}

Es bastante curioso como estos pequeños huevos de pascua han generado gran controversia y curiosidad en diferentes blogs en la web con esto se puede lograr dimensionar un poco el impacto de este SO ya que logro tener gran impacto incluso en el cine 


\section*{UNIX  APLICACIÓN EN  LOS  VIDEOJUEGOS}
Las personas  que dicen que  Unix no es  un sistema operativo  para  juegos, están un poco  equivocados, la famosa consola PS4  tiene en su carrazón un sistema basado en  Unix, 
de la  rama de Berkeley, y basado en el  FreeBSD(FreeBSD 9). Esta información  parece de  otro  mundo, debido  a que  nadie  esperaba  que  una  consola  de  última generación  
pueda  estar  basada  en  Unix , cuando el  mercado de videojuegos  para  Unix/Linux  es  prácticamente inexistente para  ciertas  personas.
 
Por otra  parte  la decisión de diseño de la  PS4 en  Unix, no  debe  ser  tan sorpresiva  para  aquellas  personas  que  están  muy bien  informados  de lo que pasa  alrededor 
de los sistemas  operativos, ya  que  Sony lleva ya algunas  iteraciones con sistemas Unix.
 
La  famosa  PS3  es  una  consola  que  también  se  baso en el sistema  operativo  Unix, o más propiamente dicho  en  una  distribución  Linux llamada  Yellow Dog, que  a su vez 
basada en una distribución famosa de la época en CentOS. La consola  contaba   con una  opción que le permitía directamente  ejecutar el sistema operativo Yellow  Dog.
 
El sistema  operativo de  la consola  PS Vita  estaba  basada  en  FreeBSD, cuyo sistema tiene muchas soluciones compatibles con los  teléfonos  y las  tablets basados en 
Android.
 
Un dato importante  es  que  Sony no  fue la única  en  implementar  el sistema  operativo  Unix, Apple  también dio ese giro cuando cambio al MacOS X, creando  su versión del 
sistema operativo sobre  la  base del sistema DARWIN( o Open Darwin) una versión especial  del FreeBSD 5.0  y sobre la cual  configuro  la  interfaz gráfica que hoy en día  se 
asocia  con la  generación actual de las MacOS.

Para  los que  no  sabían  que  es  FreeBSD, es  un sistema  operativo de  Unix de la  rama de Berkeley, una de las grandes  universidades que han investigado sobre Unix, el 
sistema nación  un poco  después de  Linux   y  su enfoque  siempre  fue la estabilidad  y la seguridad, razón por la cual siempre fue  utilizado en servidores  Cliente / 
Principal, esta fue  una  razón por la que  no fue  tan conocido como  Linux, pero  aun así  fue escogido  por  Sony  y Apple ya que  es  un sistema operativo que  brinda 
robustez.


\section*{¿UNIX ES  SEGURO?}
Para  poder  hablar  de  la seguridad  en un sistema operativo, deben conocer  algunas cosas básicas  sobre la seguridad informática.

\subsection*{¿QUÉ ES SEGURIDAD?}
La seguridad  es una característica  de cualquier sistema (informático o no) que nos indica que ese sistema está  libre de todo peligro, daño o riesgo, y que es, en cierta manera 
infalible. Pero a la hora de hablar de seguridad en SO o redes de computadores, se suele hablar de fiabilidad (probabilidad de que un sistema se comporte tal y como se espera de 
él).

En la seguridad  de  los  SO, nos centramos en tres aspectos:
\begin{itemize}
	\item La confidencialidad nos dice que los objetos de un sistema han  de ser accedidos solo por elementos autorizados para ello.
	\item La integridad que los objetos solo pueden ser modificados por autorizados. 
	\item La disponibilidad que dichos elementos deben poder ser accesibles por los autorizados, contrario a la negación de servicio.
\end{itemize}

\textbf{Amenazas a  la seguridad de nuestro sistema:} 

\begin{itemize}
	\item	Interrupción del  sistema, si se queda inútil o  no disponible.
	\item Interceptación, es si un elemento no autorizado, consigue un acceso a un determinado objeto del sistema.
	\item Modificación, si además de conseguir el acceso se consigue modificar el objeto.
	\item Destrucción, es la modificación de un objeto que queda inutilizado.
	\item Fabricación, si se trata de una modificación destinada a conseguir un objeto similar al atacado de forma que sea difícil distinguir entre el original y el fabricado.
\end{itemize}

\textbf{¿De qué nos protegemos?}

\begin{itemize}
	\item Personas como son el mismo usuario, o curiosos o crackers o piratas.
	\item Amenazas lógicas, como software incorrecto, puertas traseras, bombas lógicas, canales cubiertos u ocultos, gusanos, caballos de Troya, conejos o bacterias, virus, 		
			applets hostiles.
	\item Catástrofes naturales o artificiales. 
\end{itemize}

\textbf{¿Cómo protegernos?}

La seguridad se abarca desde tres puntos:
\begin{itemize}
	\item Prevención, utilizando mecanismos de seguridad como son autenticación e identificación, control de acceso, separación por niveles a proteger y seguridad en las 
			comunicaciones.
	\item Detección, programas de auditoría.
	\item Recuperación, copias de seguridad.
\end{itemize}

\subsection*{LA SEGURIDAD  EN  UNIX}
Dentro  de  toda  la gran familia de Unix  hay  una  serie de sistemas denominados "Unix Seguros"  o "Unix Fiables" (Trusted Unix), son sistemas  que  tiene  un excelente sistema 
de control, evaluados  por la National Security Agency (NSA) estadounidense y clasificados en niveles seguros (B o A). entre estos sistemas  de  "Unix  Seguros" podemos  
encontrar  a  AT\&T System V/MLS  y  OSF/1 (B1), Trusted Xenix (B2) y  XTS-300 STOP 4.1 (B3, considerados como unos de los  sistemas  más  seguros  del  mundo segun  la  NSA).
 
La gran mayoría de los sistemas operativos de  Unix  como Solaris AIX están clasificados  como (C2), y algunos otros, como  Linux  se consideran sistemas  (C2 de facto: al  no 
tener  una empresa que pague el proceso de evaluación de la NSA, no se encuentran catalogados), aunque  puedan implementar todos  los mecanismos  de un sistema C2.


\section*{UNIX EN DISPOSITIVOS MÓVILES}
En el 2012 se reportaron 839.705 usuarios de teléfonos inteligentes (Smartphone), un 60\% más que lo del 2011, cuando llegó a 522.640 usuarios, según los últimos datos de la 
encuesta de Tecnologías de la Información y la Comunicación (TIC) del Instituto Nacional de Estadística y Censos (INEC).  Con esta cifra se puede concluir que el Smartphone es un 
dispositivo que hace parte de la vida cotidiana de una gran parte de la sociedad y que esta parte va en aumento.

Como sabemos estos dispositivos requieren de un sistema operativo que  administre de este, la tecnología desarrollo e investigación sobre estos a crecido exponencialmente creando 
gran competencia entre los fabricantes aunque cada sistema operativo móvil posee características diferentes prácticamente todos los sistemas operativos actuales se basan o 
derivan de sistemas Unix , ya que la gran mayoría de estos están basados en  modelo de capas que con lleva a una implementación de conceptos Unix.

En si podemos concluir que Unix no está implementado ni destinado totalmente a dispositivos móviles pero los sistemas operativos de que utilizan los móviles poseen algo 
relacionado con Unix y podemos decir que hay algo de Unix en la gran mayoría de estos dispositivos que cada día se hacen mas comunes y necesarios para el diario vivir


\section*{UNIX COMO FUENTE DE TRABAJO}
Las implementaciones de Unix en servidores requieren de manteniendo adaptación y un buen uso como todo software y esto es algo que no cualquier persona puede realizar por ello 
para que en un servidor de este tipo tenga un buen rendimiento es conveniente que sea administrado por alguien que conozca pueda identificar una falencia fácilmente y pueda 
optimizar el uso lo que con lleva a que empresas requieran  empleados con estas capacidades especificas generando una buena demanda por este tipo de personal el cual es bien 
remunerado a este cargo se le ha dado el nombre de ingeniero Unix y para el cual es fácil encontrar ofertas de trabajo tanto nacionales como internacionales este es un claro 
ejemplo del enorme impacto en la sociedad de este SO.


\section*{¿CUÁNTO  CUESTA  CAMBIAR  UN SISTEMA  OPERATIVO?}
Es  un tema  que  las  grandes empresas  manejan  según  la cantidad de ordenadores  y la complejidad del sistema  operativo, para  algunas   la  inversión que  tuvieron que  
hacer  es recuperada  de manera   inmediata, para  otras  empresas  el  cambio  en  los sistemas  operativos no les parece  tan  primordial.
cuando un  empresa  decide   cambiar  de un  sistema  operativo  privativo  a  un sistema  operativo libre, se  encuentran  empresas  como  Andago y Esware Linux, empresas  
españolas  que  se  centran  en  realizar evaluaciones de viabilidad técnica  y económica, después  de realizar  las  evaluaciones  pertinentes   instalan  un  software  a  
pequeña  escala, para  comprobar y  convencer  al usuario de que  funciona adecuadamente, no solo se encargan de instalar y  hacer  las  pruebas pertinentes al sistema operativo, 
si no   también  deben  de   formar  al personal  de la  empresa  y ofrecer  servicio  técnico, este  proceso  de adecuación les  toma aproximadamente   un  año,  " en este  
tiempo  las  empresas  empiezan a  recuperar  su  inversión",  dice  David Aycart, fundador  de Esware Linux.
algunos casos  notables  de empresas  que  cambiaron su sistema  operativo fueron,  el caso de Dresdner  que sustituyo sus  30 servidores Unix de 50.000 dolares cada  uno con  40 
linux de 3.000 dolares cada uno,y mientras  que  los primeros  tardaban  cerca de  17 horas en calcular determinadas operaciones, los de  linux  realizaban el trabajo en 11 
minutos aproximadamente.
otro caso  fue  el de Bussitel, una Pyme valenciana que  cuenta  con un canal de  información a bordo de los autobuses urbanos, cambio a un sistema libre debido a que necesitaba 
un software más específico, y además  termino  ahorrando  un 50\% de sus  costes informáticos.
Pero  Microsoft defiende la competitividad de Windows.diciendo que  "El precio de adquisición de las licencias supone sólo entre un 3\% y un 5\% del coste total de la solución", 
dice Isaac Hernández.y  también que  "Hay menos gastos en licencias y más en servicios", resume  Jesús Pedraza, de IDC.






 