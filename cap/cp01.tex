\chapter*{SISTEMAS OPERATIVOS MÓVILES ALTERNATIVOS}

\section*{Introducción}
Constantemente nos damos cuenta del gran impacto que tienen en la sociedad los sistemas operativos móviles, esto debido a la infinidad 
de posibilidades que permiten, desde conectividad inalámbrica hasta aplicaciones de desarrollo. Si bien la gran mayoría de los usuarios 
han optado al día de hoy por Android e iOS, los sistemas operativos de Google y Apple respectivamente, existen diversidad de SO móviles 
alternativos que no por ser los menos utilizados tienen capacidades limitadas. En este capítulo se hablará de esos SO alternativos, 
planteando cada una de sus características, sus fortalezas y debilidades, haciendo un recorrido desde aquellos que se han quedado en la 
historia, los que continúan siendo de agrado para algunos usuarios y de los que no tienen gran tiempo en el mercado o que están en fase 
de desarrollo y que podrían ser promesa en el campo de los sistemas operativos móviles.

\section*{Sistemas Operativos Móviles}
Los sistemas operativos móviles, también llamados SO móviles son sistemas operativos diseñados para proveer el control de distintos dispositivos móviles, tales como 
teléfonos móviles, smartphones, tablets, PDA’s (Personal Digital Assistant) y otros tales como relojes inteligentes, consolas portátiles, y dispositivos multimedia. Los SO 
móviles modernos combinan las características de los SO de los computadores y otras características, como el control de la pantalla táctil, las llamadas a celulares, 
bluetooth, Wi-Fi , GPS, cámara, reconocimiento de voz, acelerómetro, entre otros; lo que hace que estos SO estén orientados a la conectividad inalámbrica y a la multimedia.

\subsection*{Historia}
En el vertiginoso mundo de tecnología en el que vivimos, pasamos por alto la evolución por la que han pasado los distintos avances tecnológicos para llegar a lo que son hoy 
en día, lo que no es diferente en la tecnología movil que a pesar de ser parte importante en nuestras vidas desconocemos y de paso olvidamos la importancia de los distintos 
sistemas operativos que alguna vez existieron y ya no están con nosotros y que proporcionaron las bases de los que son hoy los SO móviles reyes del mercado. Algunos de 
estos han sabido mantenerse con el tiempo, otros ya muestran debilidades y se encuentran en el final de su vida útil ya resignados a sucumbir frente a los distintos avances 
que arrasan a cualquier intento de tecnología presente en el pasado.

Y como olvidar , por ejemplo,  a Palm OS sistema operativo precursor de tantos avances tecnológicos en equipos como el Palm Pilot y el Palm Zire, entre tantos otros 
dispositivos fabricados por Palm, una empresa que hizo historia y que ahora HP, luego de adquirirla a mediados del año pasado, ha pasado a dejarla en segundo plano. Todo 
hace pensar que Palm ya no existe más.

Symbian, sistema operativo nacido el año 2001 al alero de una exitosa Nokia — por esos años líder indiscutido del mercado — vio su debut en su versión 6.0 y 6.1 
(anteriormente era llamado EPOC) en el Nokia 9210 Communicator, un equipo revolucionario en esos tiempos, era el único capaz de enviar y recibir Fax.

Casi 500.000 equipos Symbian fueron fabricados ese 2001, el año siguiente ya eran 2.100.000 equipos. En ese tiempo nació también la UI de la Serie 60, posteriormente S60. 
El resto es historia. Nokia ya le puso certificado de muerte al exitoso Symbian luego de su asociación con Microsoft.
Intel sigue intentando sacarlo a flote, MeeGo ya fue desahuciado por Nokia, tuvo una vida muy corta y ningún equipo en el mercado.
Dentro de las alternativas fuertes es todo mucho más claro, lo reyes actuales son Android, iOS y Blackberry OS, un poco más atrás vienen competidores buscando su lugar en 
la torta, Blackberry OS, Firefox OS y Windows Phone  – sucesor por excelencia del ya desaparecido Windows Mobile. El legado de algunos es innegable, Maemo dio paso a MeeGo, 
podríamos decir que Palm OS le heredó en parte su legado a webOS y Windows Mobile le cedió su puesto a Windows Phone 7.
En términos generales podríamos decir que han existido 16 plataformas principales para móviles, 10 de las cuales están aún (algunas pocas estaran, es el caso de QNX) 
compitiendo en el mercado.
Dicen que es importante conocer la historia para no volver a cometer los mismo errores, sin embargo, también es importante saber adaptarse al mercado y a las exigencias de 
los usuarios, en caso contrario se corre el riesgo de caer en el olvido y en el desuso, varios de estos sistemas operativos pueden dar cuenta de ello.

Los siguientes hitos de los sistemas operativos móviles reflejan el desarrollo en teléfonos móviles y smartphones:
\begin{itemize}
	\item 1979-1992 Los teléfonos móviles utilizan sistemas integrados para controlar la operación.
	\item 1993 El primer teléfono inteligente , el IBM Simon , posee una pantalla táctil, correo electrónico y funciones de PDA.
	\item 1996 Palm Pilot 1000 asistente personal digital (PDA) se introduce con el sistema operativo Palm OS.
	\item 1996 Se introduce el primer Windows CE para dispositivos de bolsillo.
	\item 1999 SO Nokia S40 se introduce oficialmente junto con el Nokia 7110.
	\item 2000 Symbian se convierte en el primer sistema operativo móvil moderno en un teléfono inteligente con el lanzamiento del  Ericsson R380.
	\item 2001 El Kyocera 6035 es el primer teléfono inteligente con Palm OS.
	\item 2002 Windows CE ( Pocket PC ) se introducen a los teléfonos inteligentes.
	\item 2002 BlackBerry lanza su primer teléfono inteligente.
	\item 2005 Nokia introduce Maemo OS en la primera tablet con internet, la N770.
	\item 2007 Sale al mercado el iPhone de Apple con el sistema operativo iOS, un dispositivo móvil con comunicación a internet.
	\item 2007 Se forma la Open Handset Alliance (OHA) por Google , HTC , Sony , Dell , Intel , Motorola , Samsung , LG etc.
	\item 2008 OHA lanza Android 1.0 con el HTC Dream (T-Mobile G1) como el primer teléfono con Android.
	\item 2009 Palm presenta webOS con el Palm Pre . Pero para el 2012 dispositivos con este sistema operativo se dejan de vender.
	\item 2009 Samsung anuncia el Bada OS con la introducción del Samsung S8500.
	\item 2010 Windows Phone OS es publicado, pero no es compatible con el anterior Windows Mobile OS.
	\item 2011 MeeGo el primer SO móvil Linux , combinando Maemo y Moblin , se introduce con el Nokia N9 , en colaboración con Nokia , Intel y la Fundación Linux.
	\item En septiembre de 2011 Samsung, Intel y la Fundación Linux anunciaron que sus esfuerzos pasarían de Bada y MeeGo para Tizen durante 2011 y 2012.
	\item En octubre de 2011 el proyecto de Mer se anunció, en torno a un ultra-portátil Linux + HTML5 / QML / JavaScript Core para la construcción de los productos con, 	
			derivadas de la base de código de MeeGo. 
	\item 2012 Mozilla anunció en julio de 2012, que el proyecto anteriormente conocido como "Boot to Gecko" era ahora Firefox OS contó con la colaboración de varios 		
			fabricantes de equipos móviles.
	\item 2013 Canonical anunció Ubuntu Touch , una versión de la distribución de Linux diseñada expresamente para los teléfonos inteligentes. El sistema operativo se basa 
			en el kernel Linux de Android, usando los controladores de Android, pero sin usar código Java como en Android.
	\item 2013 BlackBerry lanzó su nuevo sistema operativo para teléfonos inteligentes y tabletas, BlackBerry 10.
\end{itemize}

\subsection*{Palm OS}
Palm OS es un sistema operativo propietario destinado a dispositivos móviles, más especificamente a PDAs (Personal Digital Assistant). Palm OS comenzó su desarrollo en 1996 y Palm Inc. comenzó a licenciarlo en diciembre de 1997 con sus novedosos aparatos PalmPilot.                                               

A partir de ese momento el soporte y el desarrollo de Palm OS se disparó, llegando en enero del 2001 a tener 100.000 personas registradas en su 
red de desarrolladores trabajando en proyectos para Palm OS. Palm OS fue uno de los pioneros en el mercado de los dispositivos móviles y por 
varios años se mantuvo como uno de los mejores sistemas operativos, sobre todas las cosas por ser muy usable y simple.                                                                                                                                                                                    
                                                                                                                                                                              
Las primeras versiones de este sistema operativo estuvieron basadas en un SO multitareas creado por Motorola. Las principales características de la plataforma Palm eran:  
\begin{itemize}
	\item Hardware altamente integrado con el SO, basado en un procesador de 68k.
	\item Usaba un display monocromático; preferible antes que implementar los colores de manera pobre.
	\item Pocas funciones del SO, se centraba sobre todo en la usabilidad.
	\item Estaba diseñado para ser una herramienta práctica, no un sistema orientado a personas con conocimiento informático.	
\end{itemize}                                                                         

Características:

\begin{itemize}
	\item Arquitectura basada en procesadores ARM de 32 bits.
	\item Soporte para tamaño de pantalla hasta 320x480.
	\item Soporte multilenguaje, japonés y chino simplificado.
	\item Menos de 300k solo para el SO (RAM)
	\item Máximo de 128 MB de RAM
\end{itemize}

\subsection*{Windows Phone}
Windows Mobile es un SO de la familia Windows CE, desarrollado por Microsoft .A pesar de llevar el nombre Windows, no es un sistema derivado ni es una versión recortada del 
mismo, sino que es un nuevo sistema diseñado específicamente para dispositivos móviles.

Los primeros dispositivos que se comenzaron a fabricar con lo que sería el sistema Windows Mobile datan del año 2000. Para ese entonces, fue lanzado como Pocket PC 2000 y 
estaba basado en Windows CE 3.0.

Este sistema, está estrechamente vinculado a otros productos de la misma marca (servicios Live, Office Mobile, Internet Explorer Mobile, etc.) y cuenta con una interfaz 
gráfica de muy buena calidad, y muy similar a la de los sistemas operativos Windows.

Ambas cosas, ayudan a disminuir la curva de aprendizaje de los usuarios pues proveen un entorno de trabajo muy similar al que se tiene en el hogar o en la oficina.

\subsubsection*{Kernel Unificado}
\begin{itemize}
	\item El kernel de Windows CE puede manejar mas de 32000 procesos simultáneos, cada uno con 2GB de memoria virtual compartida.
	\item El filesystem soporta archivos de hasta 4GB y encriptación de dispositivos de almacenamiento externo.
\end{itemize}

\subsubsection*{Variadas Arquitecturas}
Trabaja con procesadores de arquitecturas x86, ARM, SH4 y MIPS.

\subsubsection*{Sistema de Tiempo Real}
\begin{itemize}
	\item Interrupciones anidadas.
	\item Quantums de tiempo por hilo de ejecución.
	\item 256 niveles de prioridad para hilos de ejecución.
\end{itemize}

\subsubsection*{Código Compartido}
El kernel de Windows CE es, a partir de la última versión (6.0) 100\% código compartido. Lo que comprende según Microsoft, unas 3,9 millones de líneas de código.

\subsubsection*{Características de Seguridad:}
\begin{itemize}
	\item Protección del dispositivo con contraseña.
	\item Control de acceso con contraseña al sincronizar con un PC.
	\item Aumento exponencial del tiempo de espera tras intento de acceso incorrecto.
	\item Formateo remoto del dispositivo para prevenir el acceso no autorizado a información.
	\item Cifrado del contenido de la tarjeta extraíble para prevenir el acceso no autorizado a información.
	\item Cifrado en SSL para datos transmitidos entre el dispositivo y el servidor de correo corporativo.
	\item Uso de estándar AES 128 y 256 para cifrado en comunicaciones SSL.
	\item El modo Bluetooth visible (discoverable) del dispositivo puede denegarse para prevenir la seguridad.
	\item El control de ejecución de aplicaciones permite bloquear la ejecución de aplicaciones no firmadas.
	\item Permitir o bloquear la ejecución de aplicaciones y librerías DLL no firmadas.   
\end{itemize}
Actualmente, este sistema se encuentra en una buena posición en el mercado, ganando terreno lentamente. Más específicamente, Microsoft tuvo un total de 12\% del mercado 
entre PDAs y smartphones en el primer cuarto de 2006. En primer lugar estuvo Symbian (54,4\%) y le siguió Linux con un 21,8\%.La última versión de este sistema es la 
versión 6.1, que fue una actualización menor, desde la anterior versión estable, la 6.0.

\section*{Blackberry OS}
El sistema operativo Blackberry fue desarrollado especialmente para dispositivos  móviles el cual en sus inicios solo eran disponible para teléfonos inteligentes 
"smartphones". El sistema operativo proporciona multitarea y es compatible con dispositivos de entrada especializados que han sido adoptadas por BlackBerry Ltd.

\subsection*{Características:}
El SO BlackBerry está claramente orientado a su uso profesional como gestor de correo electrónico y agenda. Desde la cuarta versión se puede sincronizar el dispositivo con el correo electrónico, el calendario, tareas, notas y contactos de Microsoft Exchange Server además es compatible también con Lotus Notes y Novell GroupWise.

BlackBerry Enterprise Server (BES) proporciona el acceso y organización del email a grandes compañías identificando a cada usuario con un único BlackBerry PIN. Los usuarios más pequeños cuentan con el software BlackBerry Internet Service, programa más sencillo que proporciona acceso a Internet y a correo POP3 / IMAP / Outlook Web Access sin tener que usar BES.

Al igual que en el SO Symbian desarrolladores independientes también pueden crear programas para BlackBerry pero en el caso de querer tener acceso a ciertas funcionalidades restringidas necesitan ser firmados digitalmente para poder ser asociados a una cuenta de desarrollador de RIM.



\section*{Symbian OS}
Es el resultado de una alianza entre varias empresas multinacionales de renombre en el mercado tales como Nokia, Sony Ericsson, Samsung, Siemens, Motorola y otras.Sus 
orígenes provienen del EPOC32, otro sistema operativo para dispositivos móviles, el cual pertenece a una familia de sistemas operativos que tiene sus orígenes a finales de 
1980 y principios de 1990 con el EPOC16.

Luego de unos años, más precisamente en 1997, apareció la primera versión del denominado EPOC32, que luego pasaría a llamarse Symbian OS.

\subsubsection*{Características:}
Symbian OS posee un núcleo de tiempo real. 

Es un sistema operativo con un microkernel y capacidad multithreading. 

Soporta las arquitecturas de los últimos CPU e incluso soporta hardware single-chip" o de un solo chip.

Cuenta con un sistema de archivos de alta performance que soporta las últimas memorias NOR, NAND, SD y MMC.

Las versiones 9.3, 9.4 y 9.5 (última versión), soportan paginación bajo demanda, una característica de la que se enorgullece mucho la compañía. La paginación bajo demanda 
permite un mejor aprovechamiento de la memoria RAM de los dispositivos a que solo se carga en memoria la "página" que se va a ejecutar.

Entre los servicios genéricos que brinda el SO, se encuentran una base de datos SQL, seguridad integrada contra malware y viruses y soporte para varias plataformas de 
desarrollo como C++, J2ME, C y MIDP 2.0.

En la actualidad, la multinacional Nokia es la que provee mayor cantidad de dispositivos móviles equipados con Symbian, seguida por Sony Ericsson, Motorola, Samsung, 
Panasonic y otros. Symbian continúa innovando en el mercado de las comunicaciones móviles con tecnologías de última generación.

\section*{Firefox OS}
\section*{Ubuntu Touch}
\section*{Otros}








