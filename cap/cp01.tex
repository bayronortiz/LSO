\chapter{SISTEMAS OPERATIVOS MÓVILES ALTERNATIVOS}

\section*{Introducción}
Constantemente nos damos cuenta del gran impacto que tienen en la sociedad los sistemas operativos móviles, esto debido a la infinidad de posibilidades que permiten, desde conectividad inalámbrica hasta aplicaciones de desarrollo. Si bien la gran mayoría de los usuarios han optado al día de hoy por Android y iOS, los sistemas operativos de Google y Apple respectivamente, existen diversidad de SO móviles alternativos que no por ser los menos utilizados tienen capacidades limitadas. En este capítulo se hablará de esos SO alternativos, planteando cada una de sus características, sus fortalezas y debilidades, haciendo un recorrido desde los SO móviles que se han quedado en la historia, los que continúan siendo de agrado para algunos usuarios y de los no tienen gran tiempo en el mercado o que están en fase de desarrollo y que podrían ser promesa en el campo de los sistemas operativos móviles.

\section*{Sistemas Operativos Móviles}
\subsection*{Palm OS}
Palm OS es un sistema operativo propietario destinado a dispositivos móviles, más especificamente a PDAs (Personal Digital Assistant). Palm OS comenzó su desarrollo en 1996 y Palm Inc. comenzó a licenciarlo en diciembre de 1997 con sus novedosos aparatos PalmPilot.                                               

A partir de ese momento el soporte y el desarrollo de Palm OS se disparó, llegando en enero del 2001 a tener 100.000 personas registradas en su 
red de desarrolladores trabajando en proyectos para Palm OS. Palm OS fue uno de los pioneros en el mercado de los dispositivos móviles y por 
varios años se mantuvo como uno de los mejores sistemas operativos, sobre todas las cosas por ser muy usable y simple.                                                                                                                                                                                    
                                                                                                                                                                              
Las primeras versiones de este sistema operativo estuvieron basadas en un SO multitareas creado por Motorola. Las principales características de la plataforma Palm eran:  
\begin{itemize}
	\item Hardware altamente integrado con el SO, basado en un procesador de 68k.
	\item Usaba un display monocromático; preferible antes que implementar los colores de manera pobre.
	\item Pocas funciones del SO, se centraba sobretodo en la usabilidad.
	\item Estaba diseñado para ser una herramienta práctica, no un sistema orientado a personas con conocimiento informático.	
\end{itemize}                                                                         

Características:

\begin{itemize}
	\item Arquitectura basada en procesadores ARM de 32 bits.
	\item Soporte para tamaño de pantalla hasta 320x480.
	\item Soporte multilenguaje, japonés y chino simplificado.
	\item Menos de 300k solo para el SO (RAM)
	\item Máximo de 128 MB de RAM
\end{itemize}

\subsection*{Windows Phone}
Windows Mobile es un SO de la familia Windows CE, desarrollado por Microsoft .A pesar de llevar el nombre Windows, no es un sistema derivado ni es una versión recortada del mismo, sino que es un nuevo sistema diseñado específicamente para dispositivos móviles.

Los primeros dispositivos que se comenzaron a fabricar con lo que sería el sistema Windows Mobile datan del año 2000. Para ese entonces, fue lanzado como Pocket PC 2000 y estaba basado en Windows CE 3.0.

Este sistema, está estrechamente vinculado a otros productos de la misma marca (servicios Live, Office Mobile, Internet Explorer Mobile, etc.) y cuenta con una interfaz gráfica de muy buena calidad, y muy similar a la de los sistemas operativos Windows.

Ambas cosas, ayudan a disminuir la curva de aprendizaje de los usuarios pues proveen un entorno de trabajo muy similar al que se tiene en el hogar o en la oficina.

\subsubsection*{Kernel Unificado}
\begin{itemize}
	\item El kernel de Windows CE puede manejar mas de 32000 procesos simultáneos, cada uno con 2GB de memoria virtual compartida.
	\item El filesystem soporta archivos de hasta 4GB y encriptación de dispositivos de almacenamiento externo.
\end{itemize}

\subsubsection*{Variadas Arquitecturas}
Trabaja con procesadores de arquitecturas x86, ARM, SH4 y MIPS.

\subsubsection*{Sistema de Tiempo Real}
\begin{itemize}
	\item Interrupciones anidadas.
	\item Quantums de tiempo por hilo de ejecución.
	\item 256 niveles de prioridad para hilos de ejecución.
\end{itemize}

\subsubsection*{Código Compartido}
El kernel de Windows CE es, a partir de la última versión (6.0) 100\% código compartido. Lo que comprende según Microsoft, unas 3,9 millones de líneas de código.

\subsubsection*{Características de Seguridad:}
\begin{itemize}
	\item Protección del dispositivo con contraseña.
	\item Control de acceso con contraseña al sincronizar con un PC.
	\item Aumento exponencial del tiempo de espera tras intento de acceso incorrecto.
	\item Formateo remoto del dispositivo para prevenir el acceso no autorizado a información.
	\item Cifrado del contenido de la tarjeta extraíble para prevenir el acceso no autorizado a información.
	\item Cifrado en SSL para datos transmitidos entre el dispositivo y el servidor de correo corporativo.
	\item Uso de estándar AES 128 y 256 para cifrado en comunicaciones SSL.
	\item El modo Bluetooth visible (discoverable) del dispositivo puede denegarse para prevenir la seguridad.
	\item El control de ejecución de aplicaciones permite bloquear la ejecución de aplicaciones no firmadas.
	\item Permitir o bloquear la ejecución de aplicaciones y librerías DLL no firmadas.   
\end{itemize}
Actualmente, este sistema se encuentra en una buena posición en el mercado, ganando terreno lentamente. Más específicamente, Microsoft tuvo un total de 12\% del mercado entre PDAs y smartphones en el primer cuarto de 2006. En primer lugar estuvo Symbian (54,4\%) y le siguió Linux con un 21,8\%.La última versión de este sistema es la versión 6.1, que fue una actualización menor, desde la anterior versión estable, la 6.0.

\section*{Blackberry OS}

\section*{Symbian OS}
Es el resultado de una alianza entre varias empresas multinacionales de renombre en el mercado tales como Nokia, Sony Ericsson, Samsung, Siemens, Motorola y otras.Sus orígenes provienen del EPOC32, otro sistema operativo para dispositivos móviles, el cual pertenece a una familia de sistemas operativos que tiene sus orígenes a finales de 1980 y principios de 1990 con el EPOC16.

Luego de unos años, más precisamente en 1997, apareció la primera versión del denominado EPOC32, que luego pasaría a llamarse Symbian OS.

\subsubsection*{Características:}
Symbian OS posee un núcleo de tiempo real. 

Es un sistema operativo con un microkernel y capacidad multithreading. 

Soporta las arquitecturas de los últimos CPU e incluso soporta hardware single-chip" o de un solo chip.

Cuenta con un sistema de archivos de alta performance que soporta las últimas memorias NOR, NAND, SD y MMC.

Las versiones 9.3, 9.4 y 9.5 (última versión), soportan paginación bajo demanda, una característica de la que se enorgullece mucho la compañía. La paginación bajo demanda permite un mejor aprovechamiento de la memoria RAM de los dispositivos a que solo se carga en memoria la "página" que se va a ejecutar.

Entre los servicios genéricos que brinda el SO, se encuentran una base de datos SQL, seguridad integrada contra malware y viruses y soporte para varias plataformas de desarrollo como C++, J2ME, C y MIDP 2.0.

En la actualidad, la multinacional Nokia es la que provee mayor cantidad de dispositivos móviles equipados con Symbian, seguida por Sony Ericsson, Motorola, Samsung, Panasonic y otros. Symbian continúa innovando en el mercado de las comunicaciones móviles con tecnologías de última generación.

\section*{Firefox OS}
\section*{Ubuntu Touch}








