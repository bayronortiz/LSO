\chapter*{LOS VIDEOJUEGOS EN LINUX}

\section*{Introducción}
En la sociedad surgen muchas necesidades, también surgen a solucionarlos de una forma más eficaz, en un mundo laboral difícil de llevar genera mucho estrés, para los 
amantes de los videojuegos este tema les va a parecer muy interesante el tema del desarrollo de los videojuegos sobre y para el sistema operativo Linux, lo cual abarca 
muchos temas a tratar por ejemplo: los pasos, requerimientos y la tecnología necesaria para la creación, mejorar y gestionar la creación de un nuevo video juego. Para 
esto hay que comprender que hay muchos tipos de ellos, como lo son los de software libre o código abierto, juegos comerciales o los que son portados a Linux, como también 
se comprende que hay consolas para correr los videojuegos como lo es Pandora o SuperGamer.

Con el avance de la tecnología, han surgido nuevas plataformas, para esta ocasión es muy importante mencionar a Android, la cual esta tiene un núcleo basado en kernel 
Linux, pero hay que dejar claro que no tiene que ver absolutamente nada con el mercado de juegos de Android, también ha tenido muchas relaciones con otras plataformas, como 
el de MAC.

Lastimosamente el mercado de juegos para pc en sistemas operativos distintos a Windows, actualmente es muy reducido, ya que la mayoría de las empresas desarrolladoras que 
tienen un alto título en desarrollos comerciales más importantes del mundo, solamente desarrollan sobre Windows por su fácil manejo y de algún modo lo ven como reducción de 
costos.

\section*{Relatos Sobre los Videojuegos ya Existentes}
Como lo han planteado muchos contribuyentes y amantes al desarrollo del software libre, Linux ha llegado a tener un gran impacto en muchas ramas de la tecnología, es este 
relato se habla en la parte de desarrollo de videojuegos, el sistema operativo Linux hace parte de una listas más importantes de software libre para gestionar un proyecto 
interactivo y por aquello tiene muchas aplicaciones y modo de uso en la sociedad que trabajan en el desarrollo de la tecnología. Aunque han llegado a la conclusión que 
Linux no ha sacado al mercado muchos videojuegos por ciertas razones, las que más se resaltan es la del no conocer las capacidades del sistema operativo correspondiente al 
desarrollo de nuevos software sea para el campo laboral como para el campo interactivo, el otro factor importante para nombrar es el gran auge que empezó hace unos poco 
años sobre el desarrollo de videojuegos de alta gama, es decir: Excelentes gráficos, buenos soportes, asemejados muchos a la vida cotidiana y sobre todo ilustrativos.

Linux en los últimos años, se ha dedicado mucho a la creación de videos juegos con la capacidad de tener una plataforma en la red, más conocidos como los juegos de rol en 
primera persona Online. Al momento de aparecer esta nueva forma de gestionar, desarrollar y plantear un video juego de este tipo, Linux ha sacado al mercado muchos de ellos 
para consolas por medio de Steam. Para lo que no saben que es Steam: es una plataforma de distribución digital, gestión digital de derechos, comunicación y servicios 
multijugador que fue desarrollada por la empresa Valve Corporation. Es utilizado por pequeños desarrolladores independientes como también lo usan grandes empresas y 
corporaciones de software para la distribución de videos juegos y material multimedia relacionado en este campo.

Los videos juegos ya existentes, en la actualidad a pesar de tener un gran tiempo de ser desarrollados, por su gran auge, comentarios y excelentes puntos de vista de los 
gamers han estado en la lista de poder ser actualizados, sacar una segunda e incluso tercera parte, entre muchos otros factores.

Unas de las ventajas más importantes e impactantes que tiene Linux respecto a los videos juegos es que mientas estos en Windows pueden llegar a costar decenas e incluso 
cientos de dólares para poder adquirirlo, en Linux sencillamente son gratis, bajo ninguna circunstancias o requerimientos, el único requisito primordial es que sean 
ejecutados en este mismo sistema operativo, pero aun así la gran cantidad de videojuegos desarrollados para Windows pueden llegar a tener más impacto, lo que lleva a que la 
comunidad gamers que solo se dedica al uso de Linux les pueda parecer más atractivo lo que crea Windows y migre a su uso.

\section*{Impacto en la Sociedad}
Una de las razones por las cuales la gente normalmente no usa Linux es por sus limitaciones a nivel de videojuegos, si, esta es una de las causas del porque los usuarios no 
se cambian definitivamente a Linux y deciden conservar una partición con Windows para su diversión y entretenimiento, pero al paso de los años se han visto grandes avances 
en este aspecto, podríamos hablar de videojuegos como Lugaru ó Neverwinter Nights que para suerte de los amantes a los videojuegos de rol en línea este fue parchado para el 
sistema operativo del famoso pingüino, tuvo un muy buen recibimiento por parte de los usuarios ya que dicho parche crea una especie de instalación nativa para Linux y 
genera un buen funcionamiento a la hora de jugarlo. Pero también podemos hablar de cómo jugar videojuegos comerciales en Linux y quitar esa barrera que separa el uso de 
Windows para los videojuegos que tanto nos llaman la tensión, nos referimos a opciones como WINE y PLAY ON LINUX.

WINE es una  re implementación de la interfaz de programación de aplicaciones de Win16 y Win32 para sistemas operativos basados en Unix, no podríamos decir que es un simple 
emulador de Windows para Linux, es mejor referirnos como “Una aplicación creada por ingeniería inversa” que nos permite tener una especie de mini sistema Windows en nuestro 
Linux y podamos aprovecharlo no solo para los videojuegos, sino para muchas aplicaciones de este SO que lleguemos a necesitar. WINE es una herramienta a la que le podemos 
sacar mucho provecho pero si queremos enfocarnos solo en los videojuegos podemos ver más hacia PLAY ON LINUX, esta es una aplicación cuya base es WINE pero enfocada 
principalmente en ejecutar videojuegos de sistemas Windows en ambiente UNIX y GNU/LINUX, lo más interesante de esta aplicación es que basada en problemas que llega a 
generar WINE en la instalación de videojuegos, problemas que llegan a disminuir el rendimiento de la aplicación, fue creada para configurar a su aplicación madre para la 
adecuada ejecución de los videojuegos mediante scripts que modifican su comportamiento y así ofrecer una mejor ejecución, estos scripts también pueden ser creados por los 
usuarios y adicionarlos para arreglar bugs, su extensión de archivo es ".pol".

Ya que hemos hablado de cómo aun en nuestro sistema operativo basado en Linux podemos seguir jugando y frecuentando nuestros videojuegos favoritos, pero veamos mas allá, en 
la raíz de todo, el desarrollo de estos. La pregunta es ¿Por qué grandes desarrolladores no diseñan videojuegos para Linux?... Bueno a respuesta a esta pregunta que tantos 
se hacen y muy pocos se atreven a responder podemos plantear dos aspectos.

Aspectos Técnicos: Quizá una de las razones por las cuales grandes desarrolladores y empresas que se dedican a la producción de videojuegos de gran impacto como EA, Bizzard 
y mas, es que los controladores libres para las tarjetas graficas no son competentes para videojuegos, podrá verse como una declaración un poco absurda pero tiene algo 
de razón, ya que los fabricantes de tarjetas graficas no han llegado a liberar totalmente sus códigos para que usuarios creen controladores competentes y libres. También 
podemos referirnos a un punto muy discutido en la Web es que hay librerías que no son compatibles con Linux por problemas de licencia o porque no están portadas, pero 
también es un punto refutable ya que casi que cualquier cosa puede ser portada a Linux y más si es en base a C o C++ y mas hablando de videojuegos ya que estos son los 
mejores y más usados en la creación de videojuegos claro está sin desprestigiar a Python que últimamente ha ganado buen terreno en este tipo de     desarrollo,yapor último 
el tema de la licencia de estas seria un tema a tratar y posiblemente a solucionar. Como análisis final diríamos que en aspectos técnicos si existen razones por las cuales 
podríamos decir que no se desarrollan gran cantidad de videojuegos para Linux, pero también son razones solucionables pero eso es un tema que no vamos a tratar.

Aspectos no técnicos:     Vamos a mencionar 2 de las posibles razones que creemos que afectan este tema y las discutiremos brevemente.

"No existen jugadores en potencia":En respuesta a esto podríamos decir MENTIRA, sabemos que muchos "geeks-gamers" harían del uso de Linux totalmente si la producción 
de videojuegos aumentara para este, y a quien no le gustaría tener los videojuegos de última generación en nuestro sistema de pingüino favorito?...

La expresión "Si no son videojuegos libres los usuarios no los jugarán": Bajo a esto diríamos que es algo un poco absurdo, si sabemos que Linux es sinónimo de libertad, 
pero también tengamos en cuenta, nadie se gana la vida regalando su trabajo y todos los amantes de los  videojuegos de grandes empresas pagarían por tenerlos en su     
distribución favorita de Linux.

Ya para finalizar mencionemos, acorde avanza el tiempo hemos visto que el desarrollo de videojuegos sobre y para Linux ha crecido considerablemente, ya el pequeño catalogo 
que conocíamos de pocos videojuegos se ha extendido y tiene mucha variedad, poco a poco veremos qué futuro depara para nuestro sistema Linux y los videojuegos que tanto nos 
apasionan y de seguro tendremos grandes evoluciones en este tema y así como dijo Gabe Newell, co-fundador de Valve en el discurso inaugural de la LINUXCON en octubre de 
2013 \textit{"Es gracioso venir aquí y deciros a vosotros que Linux y el Open Source son el futuro del videojuego. Es algo así como ir a Roma y enseñarle Catolicismo al 
Papa."}, esto nos deja claro, Linux tiene mucho futuro a con los videojuegos.
