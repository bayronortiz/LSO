\chapter*{LOS VIDEOJUEGOS EN LINUX}

\section*{INTRODUCCIÓN}
En la sociedad surgen muchas necesidades, sea por razones
personales o laborales, a consecuencia de ello se tratan de
solucionar de una forma más eficaz, en un mundo laboral difícil
de llevar genera mucho estrés, para los amantes de los
videojuegos este tema les va a parecer muy interesante el tema
del desarrollo de los videojuegos sobre y para el sistema
operativo Linux, lo cual abarca muchos opciones a tratar por
ejemplo: los pasos, requerimientos y la tecnología necesaria
para la creación, mejorar y gestionar el desarrollo de un nuevo
video juego. Para esto hay que comprender que hay muchos
tipos de software libre o código abierto, juegos comerciales o los
que son portados a Linux, como también se comprende que hay
consolas para correr los videojuegos como lo es Pandora o
SuperGamer.

Con el avance de la tecnología, han surgido nuevas plataformas,
para esta ocasión es muy importante mencionar a Android, la
cual esta tiene un núcleo basado en kernel Linux, pero hay que
dejar claro que no tiene que ver absolutamente nada con el
mercado de juegos de Android, también ha tenido muchas
relaciones con otras plataformas, como el de MAC.

Lastimosamente el mercado de juegos para pc en sistemas
operativos distintos a Windows, actualmente es muy reducido,
ya que la mayoría de las empresas desarrolladoras que tienen un
alto título en desarrollos comerciales más importantes del
mundo, solamente desarrollan sobre Windows por su fácil
manejo y de algún modo lo ven como reducción de costos.
Pero también se pueden resaltar muchos puntos buenos en el
desarrollo de videojuegos en el sector económico y que de
muchas formas ha levantado este sector a base de tecnología
interactiva. Muchos críticos empresariales han tomado este
nuevo impacto y usarlo como una herramienta para el
desarrollo de la sociedad, de esta forma se plantea y se
desarrolla aplicaciones para el sector de la educación, promover
simulaciones empresariales, así muchas personas que tienen
mucho interés en los videojuegos, no solo para programadores si
no personas que desean obtener un conocimiento previo para
llegar al punto de realizar uno de ellos, verán herramientas que
tendrán la posibilidad de crear juegos complejos sin utilizar ni
una sola línea de código.

\section*{RELATOS SOBRE VIDEOJUEGOS YA EXISTENTES}

Como lo han planteado muchos contribuyentes y amantes al
desarrollo del software libre, Linux ha llegado a tener un gran
impacto en muchas ramas de la tecnología, es este relato se
habla en la parte de desarrollo de videojuegos, el sistema
operativo Linux hace parte de una listas más importantes de
software libre para gestionar un proyecto interactivo y por
aquello tiene muchas aplicaciones y modo de uso en la sociedad
que trabajan en el desarrollo de la tecnología. Aunque han
llegado a la conclusión que Linux no ha sacado al mercado
muchos videojuegos por ciertas razones, las que más se resaltan
es la del no conocer las capacidades del sistema operativo
correspondiente al desarrollo de nuevos software sea para el
campo laboral como para el campo interactivo, el otro factor
importante para nombrar es el gran auge que empezó hace unos
poco años sobre el desarrollo de videojuegos de alta gama, es
decir: Excelentes gráficos, buenos soportes, muy similares a la vida real y sobre todo ilustrativos.

Linux en los últimos años, se ha dedicado mucho a la creación
de videojuegos con la capacidad de tener una plataforma en la
red, más conocidos como los juegos de rol en primera persona
Online. Al momento de aparecer esta nueva forma de gestionar,
desarrollar y plantear un videojuego de este tipo, Linux ha
sacado al mercado muchos de ellos para consolas por medio de
Steam. Para lo que no saben que es Steam: es una plataforma de
distribución digital, gestión digital de derechos, comunicación y
servicios multijugador que fue desarrollada por la empresa Valve
Corporation. Es utilizado por pequeños desarrolladores
independientes como también lo usan grandes empresas y
corporaciones de software para la distribución de videojuegos y
material multimedia relacionado en este campo.

Los videos juegos ya existentes, en la actualidad a pesar de
tener un gran tiempo de ser desarrollados, por su gran auge,
comentarios y excelentes puntos de vista de los gamers han
estado en la lista de poder ser actualizados, sacar una segunda e
incluso tercera parte, entre muchos otros factores.

Unas de las ventajas más importantes e impactantes que tiene
Linux respecto a los videojuegos es que mientas estos en
Windows pueden llegar a costar decenas e incluso cientos de
dólares para poder adquirirlo, en Linux sencillamente son gratis,
bajo ninguna circunstancias o requerimientos, el único requisito
primordial es que sean ejecutados en este mismo sistema
operativo, pero aun así la gran cantidad de videojuegos
desarrollados para Windows pueden llegar a tener más impacto,
lo que lleva a que la comunidad gamers que solo se dedica al uso
de Linux les pueda parecer más atractivo lo que crea Windows y
migre a su uso.

\section*{IMPACTO EN LA SOCIEDAD}

Una de las razones por las cuales la gente normalmente no usa
Linux es por sus limitaciones a nivel de videojuegos, sí, esta es
una de las razones por las que los usuarios no se cambian
definitivamente a Linux y deciden conservar una partición con
Windows para su diversión y entretenimiento, pero al paso de
los años se han visto grandes avances en este aspecto, existen
videojuegos como Lugaru ó Neverwinter Nights que para suerte
de los amantes a los videojuegos de rol en línea fue parchado
para el sistema operativo del famoso pingüino, tuvo un muy
buen recibimiento por parte de los usuarios ya que dicho parche
crea una especie de instalación nativa para Linux y genera un
buen funcionamiento a la hora de jugarlo. Pero también
podemos hablar de cómo jugar videojuegos comerciales en
Linux y quitar esa barrera que separa el uso de Windows para
los videojuegos que tanto nos llaman la atención, nos referimos
a opciones como WINE y PLAY ON LINUX.

WINE es una re implementación de la interfaz de programación
de aplicaciones de Win16 y Win32 para sistemas operativos
basados en Unix, no podríamos decir que es un simple emulador
de Windows para Linux, es mejor referirnos como “Una
aplicación creada por ingeniería inversa” que nos permite tener
una especie de mini sistema Windows en nuestro Linux y que
podamos aprovecharlo no solo para los videojuegos, sino para
muchas aplicaciones de este SO que lleguemos a necesitar.
WINE es una herramienta a la que le podemos sacar mucho
provecho pero si queremos enfocarnos sólo en los videojuegos
podemos ver más hacia PLAY ON LINUX, esta es una aplicación
cuya base es WINE pero enfocada principalmente en ejecutar
videojuegos de sistemas Windows en ambiente UNIX y
GNU/LINUX, lo más interesante de esta aplicación es que
basada en problemas que llega a generar WINE en la instalación
de videojuegos, problemas que llegan a disminuir el rendimiento
de la aplicación, fue creada para configurar a su aplicación
madre para la adecuada ejecución de los videojuegos mediante
scripts que modifican su comportamiento y así ofrecer una mejor
ejecución, estos scripts también pueden ser creados por los
usuarios y adicionarlos para arreglar bugs, su extensión de
archivo es “.pol”.

Ya que hemos hablado de cómo aun en nuestro sistema
operativo basado en Linux podemos seguir jugando y
frecuentando nuestros videojuegos favoritos, pero veamos más
allá, en la raíz de todo, el desarrollo de estos. La pregunta es
¿Por qué grandes desarrolladores no diseñan videojuegos para
Linux?... Con respecto a esta pregunta que tantos se hacen y
muy pocos se atreven a responder se pueden plantear dos
aspectos.

Aspectos técnicos: Quizá una de las razones por las cuales
grandes desarrolladores y empresas que se dedican a la
producción de videojuegos de gran impacto como EA, Bizzard y
más, es que los controladores libres para las tarjetas gráficas
no son competentes para videojuegos, podrá verse como una
declaración un poco absurda pero tiene algo de razón, ya que los
fabricantes de tarjetas gráficas no han llegado a liberar
totalmente sus códigos para que usuarios creen controladores
competentes y libres. También podemos referirnos a un punto
muy discutido en la Web es que hay librerías que no son
compatibles con Linux por problemas de licencia o porque no
están portadas, pero también es un punto refutable ya que casi
que cualquier cosa puede ser portada a Linux y más si es en
base a C o C++ y más hablando de videojuegos ya que estos son
los mejores y más usados en la creación de videojuegos claro
está sin desprestigiar a Python que últimamente ha ganado buen
terreno en este tipo de desarrollo, ya por último el tema de la
licencia de estas seria un tema a tratar y posiblemente a
solucionar. Como análisis final diríamos que en aspectos técnicos
si existen razones por las cuales podríamos decir que no se
desarrollan gran cantidad de videojuegos para Linux, pero
también son razones solucionables pero eso es un tema que no
vamos a tratar.

Aspectos no técnicos: Vamos a mencionar dos de las posibles
razones que creemos que afectan este tema y las discutiremos
brevemente.
“No existen jugadores en potencia” : En respuesta a esto
podríamos decir MENTIRA, sabemos que muchos “geeksgamers”
harían del uso de Linux totalmente si la producción de
videojuegos aumentara para este, y a quien no le gustaría tener
los videojuegos de última generación en nuestro sistema de
pingüino favorito?...

La expresión “Si no son videojuegos libres los usuarios no los
jugarán”: Bajo a esto diríamos que es algo un poco absurdo, si
sabemos que Linux es sinónimo de libertad, pero también
tengamos en cuenta, nadie se gana la vida regalando su trabajo
y todos los amantes de los videojuegos de grandes empresas
pagarían por tenerlos en su distribución favorita de Linux.

Ya para finalizar mencionemos, acorde avanza el tiempo hemos
visto que el desarrollo de videojuegos sobre y para Linux ha
crecido considerablemente, ya el pequeño catálogo que
conocíamos de pocos videojuegos se ha extendido y tiene mucha
variedad, poco a poco veremos qué futuro depara para nuestro
sistema Linux y los videojuegos que tanto nos apasionan y de
seguro tendremos grandes evoluciones en este tema y así como
dijo Gabe Newell, co-fundador de Valve en el discurso inaugural
de la LINUXCON en octubre de 2013 “Es gracioso venir aquí y
deciros a vosotros que Linux y el Open Source son el futuro de
los videojuegos. Es algo así como ir a Roma y enseñarle
Catolicismo al Papa.”, esto nos deja claro, Linux tiene mucho
futuro con los videojuegos.

\section*{GRANDES AVANCES}

En los últimos años se han visto grandes acontecimientos a favor
de la industria de los videojuegos que involucran a Linux. A
finales del año 2012 y principios del 2013 aparecieron rumores y
avances, la aparición del Beta de Valve finalmente fue abierta a
todo el mundo y poco tiempo después comenzaron a verse
títulos de videojuegos muy populares ya disponibles para Linux
que podrán ser accedidos por cualquier persona que cuente con
una cuenta en Steam, además diversos estudios de juegos como
Egosoft y THQ están buscando ahora llevar sus juegos a Linux,
como un efecto dominó a causa de las actividades de Valve e
incluso se expandieron rumores en los cuales Electronics Arts
está trabajando en una estrategia sobre Linux, esto no ha
mostrado un avance inminente pero de seguro es una buena
noticia que nos permite ver el avance que se ha llevado a cabo.
También en el transcurso del 2013 se vieron avances con
respecto a algunos juegos como Unvanquished, 0 AD, y
Xonoti que se llegaron a perfilar como títulos de software libre
de alta calidad y la noticia de Blizzard y su intención de lanzar
un videojuego para Linux. Ya para el 2014 se puede ver el
progreso que se ha tenido en este campo con noticias que nos
han causado impacto y expectativa para el futuro de los
videojuegos para Linux, noticias como la liberación por parte de
Valve del código fuente de su capa de traducción gráfica entre
OpenGL y Direct3D (TOLG), que ha sido publicado en GtHubi
bajo licencia MIT, Valve al hacer esto permite a los
programadores de juegos migrar de un forma más fácil sus
proyectos hechos en Windows a otras plataformas como
SteamOS, lo mismo también es decir GNU/Linux. 
En el transcurso de este año también se han visto dos noticias
orientadas a grandes empresas desarrolladoras de videojuegos,
una de estas es Crytek, Esta famosa empresa alemana autora de
juegos como Ryse: son of Rome, Farcry y la saga de Crysis
anunció que añadirá soporte nativo Linux a su potente motor de
juego de ordenador CryEngine, la otra empresa a la que nos
referimos es GOG la compañía especialista en la venta de
videojuegos clásicos ha anunciado que tiene planes de dar
soporte a GNU/Linux, dando disponibilidad a los usuarios de
distribuciones como Ubuntu y Linux Mint ya que estas serán el
centro sus esfuerzos en compatibilidad ofreciendo un mínimo de
100 títulos de videojuegos de su catálogo.

\section*{LENGUAJES, FORMAS Y DESARROLLO DE VIDEOJUEGOS EN LINUX}

Desarrollar videojuegos en linux resulta ser un gran camino a
aquellos programadores novatos y experimentados que se
desean adentrar al mundo del desarrollo de estos, ya que en la
comunidad de usuarios de linux es muy facil encontrar ayuda y/u
orientacion, tambien la gran cantidad de librerías y software
especializado para la creacion de videojuegos que tenemos hoy
en día hace que aprender sea mucho más facil.
A continuación se verán varias opciones que a consideración de
los autores son las más conocidas y/o recomendadas para
comenzar desde lo más básico en el desarrollo de videojuegos
sobre la plataforma linux.
Comenzando desde aplicaciones básicas se pueden ver opciones
como:

\subsection*{OHRRPGCE}

El Official Hamster Republic Role Playing Game Creation Engine
es una suite Open Source “All in One” para el desarrollo de
videojuegos, esta suite fue creada inicialmente crear juegos de
genero RPG en 2d. Tiene dos opciones para que el usuario elija
como crear su videojuego, por una parte está un lenguaje propio
de la aplicacion llamado HamsterSpeak, este lenguaje es muy
simple y se puede utilizar sin necesidad de tener conocimientos
previos a la programación. Aunque también es posible
desarrollar videojuegos sin necesidad de escribir una sola línea
de código.

\subsection*{GAME EDITOR}

Game editor es una plataforma para la creacion de juegos muy
popular, es de código abierto y tiene muchas utilidades, no solo
permite crear nuestros propios juegos, también nos da la opción
de tomar un proyecto que ya haya sido desarrollado sobre esta
misma plataforma y modificar su código para adaptarlo a las
nuevas ideas que se tengan.
Game editor está disponible para las plataformas Windows, Mac
y Linux, pero no solo es posible desarrollar videojuegos para pc
también para teléfonos móviles, su interfaz es amigable con el
usuario y facil de manejar.
Ya se vieron dos muy buenas opciones para desarrollar
videojuegos de forma facil y sin tener mayor conocimientos
sobre programación. Ahora adentrémonos más en el mundo de
la programación como tal para videojuegos sobre la plataforma
linux y las opciones disponibles que se encuentran para el
desarrollo de estos.


Las siguientes herramientas se presentarán suponiendo que el
lector tiene conocimientos previos de programacion y
programacion orientada a objetos, para ser mas especificos,
tener conocimientos en lenguajes como C, C++ y Python.

\subsection*{SDL}

stas representativas siglas hacen referencia a Simple
DirectMedia Layer. SDL es una serie de librerías de desarrollo
multiplataforma por esto muy utilizada en linux aunque su
desarrollo en windows es notable, ofrece un acceso a bajo nivel
de audio, teclado, ratón y joystick, tambien un manejo de
graficos a través de OpenGL y Direct3D. SDL soporta
oficialmente Windows, Mac OS X, Linux, iOS y Android. Este
esta escrito en C y funciona de forma nativa en C++ y hay
enlaces disponibles para su uso con otros lenguajes como C#,
Python, Perl, php, java y mas.

\subsection*{PYGAME}

Pygame es un conjunto de modulos de lenguaje python
construido bajo Simple DirectMedia Layer (SDL) que nos ayuda
a la creacion de juegos y/o aplicaciones graficas en dos
dimensiones, pygame incluye graficos y bibliotecas de sonidos
diseñados para ser utilizados en python. Pygame es portable a
casi cualquier plataforma y sistema operativo, está orientado al
manejo de sprites ( Se llama sprite a cada elemento gráfico con
entidad propia y capacidad de movimiento. Por poner un
ejemplo, en un videojuego tipo matamarcianos, la nave
protagonista es un sprite, cada uno de los marcianos enemigos
también lo es, y cada bala también es un sprite. Los sprites son,
por lo tanto, elementos básicos en los videojuegos en 2D ) . El
manejo de pygame se ha hecho muy popular en el internet y
niños y adultos se divierten programando sus propios
videojuegos, esto se puede comprobar al ver la cantidad de
descargas producidas a obtener este conjunto de modulos.

\section*{EL APOYO A LOS VIDEOJUEGOS BASADOS EN LINUX}

Como se ha planteado anteriormente, el auge y éxito que ha
tenido los videojuegos realizados en Linux ha tenido un gran
impacto en la sociedad. Varios sectores de la economía se han
beneficiado con estos proyectos a largo plazo, invirtiendo un
capital limitado para promocionar los juegos y así mismo han
obtenido grandes ganancias. En el sector comercial las
pequeñas y medianas empresas dedicadas a la compra de estos
videojuegos han dado a conocer sus opiniones de las cuales un
gran porcentaje de ellas afirman que sus ventas son muy buenas
y a base a eso compran más mercancía a los respectivos
distribuidores. Ahora en el sector de la educación ha tenido una
gran influencia puesto que ha ayudado al desarrollo de la lógica
y destreza mental en los niños ya que por medios de los juegos
hacen que estudien interactivamente, pero esta parte ha tenido
muchas críticas de docentes y padres de familia que afirman que
este método de aprendizaje solo induce a la violencia y desorden
mental. Al principio se creía que esta teoría que rechazaba este
aprendizaje era algo perjudicial para los niños, pero luego
después de estudios realizados por la Universidad de Full Sail
ubicada en Estados Unidos en el estado de Florida se llegó a la
conclusión que los videojuegos influyen mucho en los niños en
crecimiento ya que aceleran su creatividad y así despiertan su
ambición por saber del porqué, cómo y el nacimiento de muchas
cosas.

A pesar que Linux no tiene muchos videojuegos, los realizados y
probados han llamado la atención de varias empresas, entre
esas esta Valve Corporation, una empresa estadounidense
dedicada al desarrollo de videojuegos, esta se hizo
mundialmente famosa por sacar al mercado su primer juego:
Half-Life y por su modificación de este mismo al reconocido
juego de Counter-Strike, un videojuego muy famoso de
jugabilidad online, es el videojuego con el más alto número de
usuarios y así convirtiéndose en el número uno en la modalidad
jugador vs jugador. Otros de sus logros fue la creación del motor
de videojuego Source que es usado para el desarrollo de las
plataformas Microsoft Windows de 32 y 64 bits, Mac OS, Linux,
Xbox, Xbox 360 y PlayStation 3, se dio a conocer en el año 2004
con el lanzamiento de Counter-Strike: Source, pero ha venido
evolucionando a medida que la competencia entre las grandes
empresas crece.

Valve Corporation al observar todo lo que Linux ha logrado con
su desarrollo, emprende un proyecto en la cual consistía en la
creación de una nueva plataforma para videojuegos onilne que
le dieron el nombre de Steam con el fin de potencializar y
ayudar a Linux en sus proyectos de desarrollo. Con este nuevo
soporte para los juegos, les ayuda mucho a sacar nuevas
actualización para ellos mismos mientras se trabajan en otros
proyectos futuros, unos de los puntos que más se destacó es que
muchas de las actualizaciones son gratuitas dependiendo del
manejo. La empresa Microsoft tuvo una breve intervención por
Linux, pero no en el sentido que le ayudó directamente, sino
porque varios accionistas de Microsoft quisieron probar suerte
con el desarrollo de Linux ya que tenía una gran base que es
Steam y aportan pequeños capitales para la financiación de
nuevos videojuegos o mejorar lo que ya estaban hechos, sin
contar con los programadores que trabajaban en Microsoft y se
pasaron a laboral en Linux. Pero aun así Linux no contaba con la
ayuda de empresas para surgir en el mercado, aquí es donde
aparece Richard Stallman, un gran programador estadounidense
y es el fundador del movimiento por el software libre en el
mundo. Richard es un personaje muy importante en la historia
de Linux a la actualidad ya que el inicio a Linux Foundation con
el fin de seguir con su movimiento y generar nuevas ideas para
la creación de software libre en el mercado tecnológico.

Linux Foundation es un consorcio tecnológico sin ánimo de lucro
establecido exclusivamente para ayudar en el crecimiento de
Linux en todos sus aspectos. La fundación nació de la unión de
los Laboratorios de desarrollo de código abierto y el Grupo de
normas gratis, estos dos consorcios tenían el mismo objetivo en
la adaptación del mercado y la estandarización de los
componentes de Hardware y Software para el sistema operativo
Linux, todo esto tuvo fecha el 21 de enero del 2007. Después de
este acto, Steam se incluyó a Linux Foundation, causó que unas
de las cabezas del equipo de Linux en Valve, Mike Sartain, unos
de los líderes de la fundición, programador a gran escala y
dueño de grandes avances de tecnología para el desarrollo de
Linux planteo las siguientes palabras:

“Unirnos a Linux Foundation es una de las muchas
formas en las que Valve están invirtiendo en el
avance de los juegos en Linux. Con estos esfuerzos
esperamos contribuir con herramientas para
desarrolladores que construyan nuevas
experiencias en Linux, hacer que los fabricantes
de hardware priorice el apoyo a Linux y en última
instancia ofrecer una plataforma elegante y
abierta a los usuarios de Linux.”

\subsection*{GUADALINEX}
\subsection*{GUADALINEX}
