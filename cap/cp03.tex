\chapter*{LINUX EN EL GOBIERNO}
Para este capítulo de Linux en el gobierno nos basaremos en aquellas naciones en las que Linux a pasado a formar parte de su estructura interna. También explicaremos las 
causas que originó este cambio y como principal consecuencia, el impacto social que este importante sistema operativo a generado a lo largo del proceso de implementación que 
adoptaron estos países.

\section*{BRASIL}
\includegraphics[scale=1]{img/cp03/img0301.png}

Tras un largo período en el que Brasil destinaba parte de su presupuesto al pago de licencias de Microsoft y Apple, optó por una alternativa que lo condujo a posicionarse 
como uno de los países pioneros en el uso de software libre. Fue en el gobierno del aún presidente Luiz Inácio Lula da Silva que se inició el cambio de Windows a Linux. La 
medida fue implementada con una gran acogida por parte de la sociedad Brasileña, ya que para la mayoría representaba una alternativa económica importante pensando en los no 
tan numerosos recursos.

Con esta puesta en escena que realizó Brasil se han logrado grandes avances en materia de desarrollo tecnológico, además de que se ha creado una cultura del libre desarrollo 
en la sociedad de este potente país, inculcando desde las aulas de clase el uso de dicho software desde muy temprana edad.

Brasil, se ha convertido en el principal país no sólo en América sino en el mundo más comprometido en la fomentación de código libre, esto le a permitido grandes ganancias ya 
que ahora no tendrá que invertir en licencias costosas como lo solía hacer con Microsoft y Apple, ahora podrá destinar estos recursos a proyectos de investigación. Con la 
implementación de este sistema operativo, se produjo en el 2008 un considerable ahorro de unos US\$ 167,8 millones de dólares. La empresa encargada de ofrecer dicha solución 
es Userful Corp, esta es una pequeña compañía canadiense encargada de desarrollar la aplicación "Multiplier", esta se ejecuta como un servicio del sistema operativo y permite 
que un computador de mesa sea compartido hasta por 10 usuarios todos conectados.

\section*{ALEMANIA}
Al igual que Brasil, Alemania y más exactamente la ciudad de Munich que utilizó en sus sistemas de redes y  de seguridad el conocido y tradicional sistema operativo de 
Microsoft ahora a optado por cambiarse a Linux, justificando como causa el "desperdicio electrónico" de ordenadores con Windows, como consecuencia de esto Munich, (la 
tercera ciudad más importante de Alemania) a destinado 30 millones de euros para tal propósito. Se puede decir que Alemania se perfila entonces a migrar posiblemente en 
todo su territorio a este sistema operativo “Linux”, ya que ademas de la evidente seguridad que este sistema operativo pueda ofrecerle gana también el país en el ámbito 
económico, por ser este, de circulación gratuita.

Para lograr una mayor aceptación por parte de la comunidad alemana, el país ofrece gratuitamente a los habitantes de Munich  miles de discos con Linux, así podrán estos 
comenzar a hacer uso del sistema operativo además de un acompañamiento que se prestara en diferentes instalaciones tales como colegios y/o universidades.

Lubuntu, "un derivado de Ubuntu", fue la distribución por la cual Munich apostó, especialmente por ser gratuito y por los costos no tan altos de memoria y eficiencia 
energética que estaban acostumbrados a gastar con Microsoft; las razones primordiales por las cuales se tuvo en cuenta en la elección de esta versión de sistema operativo 
fue que Lubuntu no presentaba problema con los requisitos de hardware, mientras que otras distribuciones si, además de la importante similitud en cuanto al entorno gráfico 
con Windows XP.

Los expertos en IT(tecnologías de la información) han expresado su voz de inconformismo ya que consideran a Windows un sistema operativo vulnerable y peligroso a la vez 
pensando en lo que pudieran llegar hacer agencias gubernamentales, si éstas tienen acceso a la información confidencial del gobierno Alemán, justificando que Microsoft 
tiene una desventaja en seguridad: su backdoor "especie de puerta trasera" éste muy seguramente podría favorecer a la NSA en tareas de espionaje, la solución a dichos 
problemas, según estos expertos, es la migración al sistema de código abierto Linux.

La administración pública alemana e IBM (empresa multinacional estadounidense de tecnología y consultoría) suscribirán un acuerdo que implica que miles de servidores serán 
migrados a la plataforma Linux.. Queda pues claro que con el pasar del tiempo y a medida que Linux ofrece el mejor desarrollo y mejoras en su sistema, se posicionará como 
uno de los favoritos en la lista de los sistemas que brindan seguridad, robustez y además a un precio muy cómodo para sus usuarios.

\section*{CUBA}
En Cuba, el acceso a una computadora resulta una tarea bastante complicada, aún más difícil resulta poder conectarse a una red, sin tener en cuenta que este país cuenta con 
un restringido ancho de banda para el acceso a internet, esto debido a la intervención de los Estados Unidos de América en la no posibilidad de tener un mayor acceso a la 
web por parte de los interesados en hacerse a estos servicios; es así pues, como resulta casi imposible el hecho de no solamente adquirir una computadora sino también el 
navegar en alguna de estas, sin antes pasar por una especie de “autorización” por parte del gobierno cubano, de la cual la gran mayoría no salen bien librada. 

Segun Alain Turiño (profesor de proyectos informaticos en escuelas cubanas) ”El primer paso para lograr una migración exitosa en un país, es que el proceso educativo en 
todos sus distintos niveles se desarrolle con software libre para que así las industrias tengan un personal calificado cuando comience un proceso de migración en las 
mismas”.

Está claro pues, que resulta muy conveniente por parte de los países interesados adoptar este tipo de iniciativas a su plan de recontextualización curricular; para así, no 
solo garantizar una familiarización, ni un óptimo manejo de un sistema operativo diferente, sino ademas, estar preparados a los distintos posibles escenarios que se 
presentan diariamente en el mundo laboral.

Es así, como ante la negativa de EE.UU para el uso de Windows de manera accesible desde todos los aspectos en Cuba, el software libre encuentra un espacio para crecer, como 
una forma de defender su soberanía informática, el gobierno cubano a buscado soluciones a estas problemáticas y así no depender de lo que pueda ofrecerle solamente el 
software privativo. 

\includegraphics[scale=1]{img/cp03/img0302.png}

“El Software libre está llamado a liderar la lucha de clases en el entorno digital para garantizar la soberanía tecnológica en América Latina”, afirmó el académico Orlando 
Cárdenas durante la XV (Dècimo Quinta)Convención y Feria Internacional Informática 2013, con sede en la capital Cubana.

Se desarrolla entonces Nova, una interesante distribución de GNU/Linux,  creada por estudiantes y profesores de la Universidad de las Ciencias Informáticas de Cuba, 
teniendo como misión facilitar la migración a código abierto. Nova, también fue implementado a causa del bloqueo que impartía EE.UU en el parque tecnológico de Cuba, y 
vieron a ésta distribución como una luz al final del túnel, la esperanza volvía.

El líder del proyecto, Angel Goñi, asegura “que el sistema permite utilizar aplicaciones modernas en una interfaz sencilla y trabajar con las máquinas obsoletas que todavía 
abundan en la isla, aunque no es "la plataforma" definitiva de Cuba para migrar a Linux”. Esta distribución, se ha convertido en un tipo de salvavidas para los habitantes 
de la isla, para así poder evadir las restricciones que les fueron implantadas por los EE.UU desde 1962, en las cuales incluye compra de programas y las actualizaciones de 
los mismos.

"Los sistemas de plataforma abierta nos permiten, en la medida que se vayan dominando todas estas técnicas y se siga profundizando en ellas, lograr una mayor inviolabilidad 
en los procesos de informática", dijo a periodistas el comandante Ramiro Valdés Menéndez, exministro del Interior del gobierno cubano.

\section*{CHINA}
Ya se ha mencionado anteriormente que la actual posicion de Microsoft se ha visto afectada, pues ha disminuido su uso en países importantes con la llegada de Linux.

Para el caso de China esto no es diferente pues, hace algún tiempo, Dell quien anunció una posible alianza con Oracle, serían los encargados de ofrecer a China un producto 
basado en Linux, todo con el fin de disminuir el creciente mercado que maneja Microsoft.

Tiempo después que el mundo informático conoció la noticia de que China incurriría en un sistema operativo como Linux para empezar a dejar a un lado el privativo sistema de 
Microsoft, las reacciones de este, no se hicieron esperar e inmediatamente tomó acciones tales como; la creacion de versiones de Windows en el idioma nativo de las 
localidades de todo el mundo.

Apesar de estos esfuerzos realizados por Microsoft, pocos años después la versión GNOME de Ubuntu no tardó en llegar para luego ser aceptada oficialmente por parte de la 
comunidad China. “Las empresas Chinas pueden llegar a ahorrarse un 70 por ciento en la compra de un servidor si escogen un servidor Dell con Base de Datos de Oracle basado 
en Linux", afirmó Foo Piau Phang, presidente de Dell en China, en un intento por erosionar la actual posición dominante del sistema operativo de Windows de Microsoft.

Es apenas lógico y natural que Microsoft se inquiete al ver que China quiera empezar grandes cambios en cuanto a seguridad informática se refiere, puesto que este país 
representa la mayor economía del mundo y a MIcrosoft le traería pérdidas millonarias. Pero es tanto el terreno que ya ha ganado Linux en esta nación, que los academicos de 
la Universidad Nacional de Defensa Tecnológica en la República Popular de China desarrollaron un sistema operativo “Kylin”, aprobado para el uso del ejército y la defensa 
nacional.

Cabe decir que este sistema operativo esta basado en Mach t FreeBSD, lo que le permite un nivel extra de seguridad al sistema operativo, y que es idéntico a (Security-
Enhnaced Linux, “Seguridad Mejorada de Linux”) que fue primeramente desarrollado por la Agencia de Seguridad Nacional de los Estados Unidos.