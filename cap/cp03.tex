\chapter*{LINUX EN EL GOBIERNO}
Para este capítulo de Linux en el gobierno nos basaremos en aquellas naciones en las que Linux a pasado a formar parte de su estructura interna. También explicaremos las 
causas que originó este cambio y como principal consecuencia, el impacto social que este importante sistema operativo a generado a lo largo del proceso de implementación que 
adoptaron estos países.

\section*{BRASIL}
\includegraphics[scale=1]{img/cp03/img0301.png}

Tras un largo período en el que Brasil destinaba parte de su presupuesto al pago de licencias de Microsoft y Apple, optó por una alternativa que lo condujo a posicionarse 
como uno de los países pioneros en el uso de software libre. Fue en el gobierno del aún presidente Luiz Inácio Lula da Silva que se inició el cambio de Windows a Linux. La 
medida fue implementada con una gran acogida por parte de la sociedad Brasileña, ya que para la mayoría representaba una alternativa económica importante pensando en los no 
tan numerosos recursos.

Con esta puesta en escena que realizó Brasil se han logrado grandes avances en materia de desarrollo tecnológico, además de que se ha creado una cultura del libre desarrollo 
en la sociedad de este potente país, inculcando desde las aulas de clase el uso de dicho software desde muy temprana edad.

Brasil, se ha convertido en el principal país no sólo en América sino en el mundo más comprometido en la fomentación de código libre, esto le a permitido grandes ganancias ya 
que ahora no tendrá que invertir en licencias costosas como lo solía hacer con Microsoft y Apple, ahora podrá destinar estos recursos a proyectos de investigación. Con la 
implementación de este sistema operativo, se produjo en el 2008 un considerable ahorro de unos US\$ 167,8 millones de dólares. La empresa encargada de ofrecer dicha solución 
es Userful Corp, esta es una pequeña compañía canadiense encargada de desarrollar la aplicacion "Multiplier", esta se ejecuta como un servicio del sistema operativo y permite 
que un computador de mesa sea compartido hasta por 10 usuarios todos conectados.

\section*{ALEMANIA}
Al igual que Brasil, Alemania y más exactamente la ciudad de Munich que utilizó en sus sistemas de redes y  de seguridad el conocido y tradicional sistema operativo de 
Microsoft ahora a optado por cambiarse a Linux, justificando como causa el "desperdicio electrónico" de ordenadores con Windows, como consecuencia de esto Munich, (la 
tercera ciudad más importante de Alemania) a destinado 30 millones de euros para tal propósito. Se puede decir que Alemania se perfila entonces a migrar posiblemente en 
todo su territorio a este sistema operativo “Linux”, ya que ademas de la evidente seguridad que este sistema operativo pueda ofrecerle gana también el país en el ámbito 
económico, por ser este, de circulación gratuita.

Para lograr una mayor aceptación por parte de la comunidad alemana, el país ofrece gratuitamente a los habitantes de Munich  miles de discos con Linux, asi podran estos 
comenzar a hacer uso del sistema operativo ademas de un acompañamiento que se prestara en diferentes instalaciones tales como colegios y/o universidades.

Lubuntu, "un derivado de Ubuntu", fue la distribución por la cual Munich apostó, especialmente por ser gratuito y por los costos no tan altos de memoria y eficiencia 
energética que estaban acostumbrados a gastar con Microsoft; las razones primordiales por las cuales se tuvo en cuenta en la elección de esta versión de sistema operativo 
fue que Lubuntu no presentaba problema con los requisitos de hardware, mientras que otras distribuciones si, ademas de la importante similitud en cuanto al entorno gráfico 
con Windows XP.

Los expertos en IT(tecnologías de la información) han expresado su voz de inconformismo ya que consideran a Windows un sistema operativo vulnerable y peligroso a la vez 
pensando en lo que pudieran llegar hacer agencias gubernamentales, si éstas tienen acceso a la información confidencial del gobierno Alemán, justificando que Microsoft 
tiene una desventaja en seguridad: su backdoor "especie de puerta trasera" éste muy seguramente podría favorecer a la NSA en tareas de espionaje, la solución a dichos 
problemas, según estos expertos, es la migración al sistema de código abierto Linux.

La administración pública alemana e IBM (empresa multinacional estadounidense de tecnología y consultoría) suscribirán un acuerdo que implica que miles de servidores serán 
migrados a la plataforma Linux.. Queda pues claro que con el pasar del tiempo y a medida que Linux ofrece el mejor desarrollo y mejoras en su sistema, se posicionará como 
uno de los favoritos en la lista de los sistemas que brindan seguridad, robustez y ademas a un precio muy cómodo para sus usuarios.