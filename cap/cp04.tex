\chapter*{IMPACTO DEL SISTEMA OPERATIVO WINDOWS EN LA SOCIEDAD (HOGAR)}

En este capítulo hablaremos del  impacto que ha tenido el sistema operativo Windows en la sociedad (hogar), y como ha venido mostrando su  huella en el mundo entero además de su 
pequeña diversidad lingüística que esta hace presencia en Windows.

\section*{INTRODUCCIÓN}
Desde la aparición de las computadoras estas han tenido un gran impacto en la sociedad, estas eran de tamaños enormes y su acceso era complicado, además han ido 
evolucionando y también sus sistemas operativos ya que ellos son los que le dan la funcionalidad al hardware que es la parte tangible de la máquina.

El avance tecnológico está transformando nuestro mundo diariamente y este ha conllevado a que las sociedades faciliten su diario vivir. Además debido a estos avances ha 
sido posible comunicarnos con amigos o seres queridos que están viviendo en otros países.

La computadora ha sido de gran ayuda en varios campos, ha funcionado como medio de comunicación como una importante herramienta de trabajo por ejemplo en el ámbito 
matemático como la ingeniería estas son útiles para el control de procesos industriales para la planificación, diseño y administración de los sistemas de instrumentación y 
automatización.
 
Además las computadoras tienen un uso popular como el uso del internet por ejemplo el uso de páginas web o redes sociales.

Son todos estos cambios los que han facilitado la vida diaria y que de alguna manera han hecho que cada uno de nosotros busque la adaptación y el manejo de nuevas 
tecnologías y de esta manera nos ha hecho depender de estas y hacerla esencial en la vida cotidiana.

A pesar de todas las ventajas y facilidades que ha puesto en nuestras manos las computadoras también es importante darnos cuenta de los problemas que estas pueden ocasionar 
o que actualmente ya están ocasionando estos son el crear desigualdades sociales, desempleo debido a que sustituyen la mano del hombre y lo orientan en emplearse solo en 
áreas técnicas, el llegar a considerar una computadora como sustituto del cerebro humano, y la creación de una dependencia importante por parte de los usuarios que 
actualmente es posible verlo como cierta adicción.


\section*{¿CÓMO SURGIÓ EL SISTEMA OPERATIVO WINDOWS?}
Las primeras computadoras que salieron manejaban el D.O.S (disk operating system ("sistema operativo de disco")) por lo cual tenías que ser un experto para manejar este 
tipo de computadoras entonces fue allí cuando un ingenioso muchacho con ganas de luchar y sobresalir en el mundo tecnológico tuvo una grandiosa idea de innovar un sistema 
operativo más fácil de manejar llamado Windows.
Estas son las versiones que se han desarrollado del sistema operativo:

\begin{description}
	\item[1985:] Windows 1.01
	\item[1986:] Windows 1.03
	\item[1987:] Windows 2.03
	\item[1988:] Windows 2.1
	\item[1990:] Windows 3.0
	\item[1992:] Windows 3.1
	\item[1992:] Windows For Workgroups 3.1
	\item[1993:] Microsoft Bob
	\item[1993:] Windows NT 3.1
	\item[1993:] Windows For Workgroups 3.11
	\item[1994:] Windows NT 3.5
	\item[1994:] Windows NT 3.51
	\item[1995:] Windows 95
	\item[1996:] Windows NT 4.0
	\item[1998:] Windows 98
	\item[2000:] Windows 2000
	\item[2000:] Windows ME
	\item[2001:] Windows XP
	\item[2003:] Windows Server 2003
	\item[2007:] Windows Vista
	\item[2009:] Windows 7
	\item[2012:] Windows 8
\end{description}


\section*{¿CÓMO HA VENIDO EVOLUCIONANDO ESTE SISTEMA OPERATIVOS?}
Este sistema operativo ha venido evolucionando por la demanda de tecnología dando lugar a que los usuarios satisfagan sus necesidades tecnológicas. 

En los años de los 70. Lo más novedoso para ese entonces hablando de sistemas era la máquina de escribir la cual todos recordamos y hasta llegamos a utilizar, si 
necesitábamos copia de un documento se utilizaba un mimeógrafo o un papel de carbón.

En este tiempo pocos habían escuchado hablar de los microequipos. Algo que estos dos jóvenes estudiantes Bill Gates y Paul Allen dos informáticos los cuales observan en 
este un futuro nuevo el cual podría evolucionar en la historia de la tecnología y se encarrilan hacia este.

En 1975 aparece una sociedad bautizada Microsoft fundada por  Bill Gates y Paul Allen. Las grandes compañías casi siempre empiezan desde lo más pequeño ellos no fueron la 
acepción. Microsoft comienza siendo pequeño pero con una gran visión: "un equipo en cada escritorio y en cada hogar". Aquí comienza una nueva forma de vida, Microsoft 
cambia nuestra forma de trabajo.

\subsection*{SURGIMIENTO DE MS-DOS}
La aparición de MS-DOS en 1980, Gates y Allen contratan al excompañero de clases de Gates de la universidad de Harvard al señor Steve Ballmer a la empresa. IBM le ofrece 
un proyecto a Microsoft cuyo nombre en código era "chess". Microsoft empieza en su renovación un nuevo sistema operativo el cual se centra en administrar y ejecutar el 
hardware de un equipo y sirve como puente entre este y los programas del equipo (Procesador de texto). En este puede ejecutarse programas informáticos el cual es bautizado 
como "MS-DOS".
 
En 1981 se introdujo al mercado el equipo de IBM con MS-DOS, el cual presento un idioma completamente nuevo al público. Escribir "C:" y otros comandos de cifrado el cual se 
convirtieron en parte del diario trabajo. También presentaron al público (\textbackslash) tecla barra diagonal invertida.
 
Este sistema operativo es eficiente pero muy complejo para muchas personas ya que este era difícil de entender. Microsoft tendría que desarrollar un mejor sistema para las 
personas si quería llegar a cada uno de los hogares del mundo.

\subsection*{1982-1985 PRESENTACION WINDOWS 1.0}
Microsoft empieza a trabajar en la primera versión de un nuevo sistema operativo su nombre de código es Interface manager considerado como su nombre final, pero aparece al 
que hoy todos conocemos como Windows por que describe mejor sus ventanas informáticas este sistemas operativo es anunciado en el año 1983 pero su desarrollo es muy lento el 
cual muchos lo llaman "Vaporware".
 
Dos años después de su anuncio más exactamente el 20 de Noviembre de 1985 Microsoft lanza Windows 1.0. Un sistema operativo totalmente renovado y mucho más accesible para 
las personas en vez de tener que utilizar comandos de MS-DOS, simplemente se  necesita mover un mouse para apuntar y hacer doble clic para abrir o seleccionar las ventanas 
que deseen. Bill Gates señala "Es un Software único, diseñado para el usuario de equipos serio".
 
Algunas de las novedades de este sistema operativo es que posee menús desplegables, barras deslizantes, iconos y cuadros de diálogo que hacen que los programas sean más 
fáciles de aprender y usar. Se puede cambiar entre varios programas sin tener que cerrar y volver a iniciar cada uno, Windows 1.0 fue presentado con varios programas entre 
los que se destacan el administrador de archivos MS-DOS, paint, windows writer, notepad, calculadora calendario, un reloj el cual fue introducido para que los usuarios 
administran sus actividades diarias y tiene además un juego  llamado Reversi. Este Sistema operativo tiene más facilidades el cual empieza a cumplir con sus objetivos que 
es llegar a tener una computadora en cada uno de los hogares ya que es práctico y fácil de usar para el usuario pero esto es solo el comienzo de Windows el cual compartirá 
con nosotros en el diario trabajo facilitándonos grandemente nuestra forma de vida.

\subsection*{1987-1990 WINDOWS 2.0-2.11 (MÁS VENTANAS, MAYOR VELOCIDAD)}
Microsoft lanza su nueva versión Windows 2.0 el 9 de diciembre de 1987, con nuevos beneficios como iconos de escritorio y con memoria ampliada. Compatibilidad y con mejores 
gráficos. Esta nueva versión nos permite intercalar ventanas controlar el diseño de la pantalla y con el teclado podemos utilizar atajos para ganar tiempo en nuestro 
trabajo. Ya en este algunos desarrolladores de software empiezan a escribir sus programas entorno a Windows.
 
La nueva versión de Windows se diseñó para el procesador Intel 286. En 1988, Microsoft se convierte en la empresa de software para equipos más grande del mundo en cuestión 
de ventas.
 
Algunos trabajadores de oficina ya utilizan los equipos como parte de su vida diaria.

\subsection*{1990-1994 WINDOWS 3.0-WINDOWS NT (SE OBTIENEN GRÁFICOS)}
Microsoft en el año 1990 más exactamente el 22 de mayo  anuncia Windows 3.0 seguido de Windows 3.1 en 1992. Estas dos ediciones venden más de 10 millones de copas en menos 
de dos años siendo el sistema operativo más vendido hasta la fecha.
 
Esta edición de Windows nos permite un mayor rendimiento a la máquina, nos regala gráficos más avanzados contenidos en 16 colores e iconos mejorados. Aparece con 
administrador de programas, administrador de archivos y administrador de impresión.
 
Windows aumenta su popularidad con el lanzamiento de su nuevo kit de desarrollo de software (SDK) de Windows. Este se usa cada vez más en cada uno de las oficinas por este 
motivo incluye juegos como Solitario, Corazones y Buscaminas. “ahora puedes usar el increíble poder de Windows 3.0 para distraerte de tus labores”.

\subsection*{WINDOWS NT}
Windows NT sale al mercado el 27 de julio de 1993, Microsoft logra un límite importante: finalización de un proyecto iniciado a finales de los 80 para desarrollar un 
sistema operativo desde el principio. A lo que Bill Gates señala: “Windows NT representa nada menos que un cambio fundamental en la forma en que las empresas pueden abordar 
sus requisitos informáticos empresariales”.
 
Este es un sistema operativo de 32 bits que lo hace una plataforma importante y compatible con programas científicos y de ingeniería superiores.

\section*{LOS BENEFICIOS QUE HA TRAÍDO ESTE SISTEMA OPERATIVO EN NUESTRA SOCIEDAD}
Este sistema posee un 90\% de participación en el mercado mundial, desconociendo otras opciones sugeridas de software libre.

Ha traído consigo mismo la viabilidad de que el  usuario pueda interactuar con este entorno grafico de una manera muy fácil  permitiéndonos comunicar con nuestros 
familiares o amigos a través del mundo, además de ser muy popular este tiene gran compatibilidad con la mayoría de  software y dispositivos que hay en el mercado.

La naturaleza de Windows  se destaca por la facilidad que tienen los principiantes al utilizar este sistema operativo.

Ha contribuido a que el individuo adquiera un nivel de racionalidad y critica al hacer uso de este sistema en donde con mayor frecuencia va incorporando a su vida diaria.

Este mundo moderno en donde el hombre hace que la tecnología está en constante evolución debido a la existencia de las computadoras.

\section*{EL FUTURO DE WINDOWS}
Muchos computadores portátiles ya no vienen con la unidad de DVD, y algunos tienen unidades de estado sólido en lugar de discos duros convencionales. Casi todo se trasmite, 
se guarda en unidades flash o se almacena en la "nube" (un espacio en línea para compartir archivos y almacenamiento).

Windows Live (un conjunto de programas y servicios gratuitos para trabajar con fotos, películas, mensajería instantánea, correo electrónico y redes sociales) está 
perfectamente integrado con Windows para que el usuario pueda mantenerse en contacto desde su equipo, teléfono o Internet a fin de extender Windows a la nube.

Planeando El futuro más inmediato en la plataforma de desarrollo de Windows.
 
Algo que ya sabíamos que iba a acabar sucediendo es que las aplicaciones de Windows 8 en ordenadores, tabletas y móviles iban a acabar compartiendo mucho más de lo que 
comparten actualmente. Y es de lo que nos acaban de hablar: de las aplicaciones universales para Windows Phone y Windows 8, tanto en tabletas como en ordenadores.

Estas aplicaciones ejecutan esencialmente el mismo código, cambiando simplemente las vistas (en definitiva, la interfaz gráfica) en función del dispositivo que las esté 
ejecutando. Esto implica, indirectamente, que las aplicaciones en Windows Phone 8.1 también pueden hacer uso de WinRT, el mismo runtime que da vida a las aplicaciones de 
Windows 8.

\subsection*{WINJS, AHORA DE CÓDIGO ABIERTO}
Además, la plataforma WinJS, pensada para desarrollar aplicaciones para Windows utilizando tecnologías Web, será multiplataforma y de código abierto a través de Microsoft 
Open Tech. Podremos crear aplicaciones para iOS, Android, Windows e incluso para la Web utilizando la misma librería.

\subsection*{WINDOWS PHONE, UNA APUESTA DE FUTURO PARA LOS FABRICANTES}
La tiranía de las cifras suele gustar a muchos, más bien los que salen ganando, e irritar a otros, los que esperaban no verlas. Pero al fin y al cabo generalmente solo se 
muestran como una fotografía fija en un instante concreto al que se le debe hacer el caso justo y oportuno. En ese sentido, el sistema operativo Windows Phone puede hacer 
todas las lecturas positivas que vea conveniente, porque las realidad de respalda.

Hasta ahora lo que hay en el mercado con Windows Phone, al margen de Nokia, no parece ni espectacular, ni especialmente llamativo. Samsung ATIV S, HTC 8X, HTC 8S, Huawei 
Ascend W1 y Huawei Ascend W2. Sin duda, no parecen el gran motivo por el cual Windows Phone haya tenido el espectacular crecimiento que ha tenido durante 2013, pasando de 
poco más de 18 millones de usuarios, a un total de 35 millones, pero habrá influido.

Pero no son los únicos. Ahora hay una nueva noticia al respecto, y aunque pueda parecer anecdótica, sin duda se puede convertir en el germen de una tendencia de la que 
habrá que estar muy atentos. Archos, el fabricante francés, ha hecho oficial que algunos de sus próximos terminales saldrán al mercado con Windows Phone. ¿Sorpresa? Como 
decimos, y los datos están para constatarlo, el sistema operativo de Microsoft ha experimentado un crecimiento, tanto cualitativo como cuantitativo, que es hace que todo 
tome un matiz distinto.

\subsection*{WINDOWS PHONE Y WINDOWS RT FUSIONADOS}
Sin embargo, es más probable que los sistemas operativos de los smartphones y las tablets se unan, quedando la mezcla resultante de la unión entre Windows Phone y Windows 
RT como único sistema operativo coexistente con Windows 8.1.

De momento toca esperar, ya que no hay nada confirmado ni oficial y sólo deseos. Eso sí, son deseos que se localizan en la parte superior de la pirámide de la compañía 
dueña de Windows, y eso hace que todos los rumores sean mucho más creíbles.