\chapter*{IMPACTO DEL SISTEMA OPERATIVO WINDOWS EN LA SOCIEDAD (HOGAR)}
\section*{Introducción}
Desde la aparición de las computadoras estas han tenido un gran impacto en la sociedad, estas eran de tamaños enormes y su acceso era complicado, además han ido 
evolucionando y también sus sistemas operativos ya que ellos son los que le dan la funcionalidad al hardware que es la parte tangible de la máquina.

El avance tecnológico está transformando nuestro mundo diariamente y este ha conllevado a que las sociedades faciliten su diario vivir.

\section*{¿Cómo Surgió el Sistema Operativo Windows?}
"Es el nombre de una familia de sistemas operativos desarrollados y vendidos por Microsoft. Microsoft introdujo un entorno operativo denominado Windows el 25 de noviembre 
de 1985 como un complemento para MS-DOS en respuesta al creciente interés en las interfaces gráficas de usuario (GUI). Microsoft Windows llegó a dominar el mercado mundial 
de computadoras personales, con más del 90\% de la cuota de mercado, superando a Mac OS, que había sido introducido en 1984."

\section*{¿Cómo Ha Venido Evolucionando Este Sistema Operativo?}
Este sistema operativo ha venido evolucionando por la demanda de tecnología dando lugar a que los usuarios satisfagan sus necesidades tecnológicas. 

En los años de los 70. Lo más novedoso para ese entonces hablando de sistemas era la máquina de escribir la cual todos recordamos y hasta llegamos a utilizar, si 
necesitábamos copia de un documento se utilizaba un mimeógrafo o un papel de carbón.

En este tiempo pocos habían escuchado hablar de los microequipos. Algo que estos dos jóvenes estudiantes Bill Gates y Paul Allen dos informáticos los cuales observan en 
este un futuro nuevo el cual podría evolucionar en la historia de la tecnología y se encarrilan hacia este.

En 1975 aparece una sociedad bautizada Microsoft fundada por  Bill Gates y Paul Allen. Las grandes compañías casi siempre empiezan desde lo más pequeño ellos no fueron la 
acepción. Microsoft comienza siendo pequeño pero con una gran visión: "un equipo en cada escritorio y en cada hogar". Aquí comienza una nueva forma de vida, Microsoft 
cambia nuestra forma de trabajo.

\section*{La Aparición de MS-DOS}
En 1980, Gates y Allen contratan al ex compañero de clases de Gates de la universidad de Harvard al señor Steve Ballmer a la empresa. IBM le ofrece un proyecto a Microsoft 
cuyo nombre en código era "chess". Microsoft empieza en su renovación un nuevo sistema operativo el cual se centra en administrar y ejecutar el hardware de un equipo y 
sirve como puente entre este y los programas del equipo (Procesador de texto). En este puede ejecutarse programas informáticos el cual es bautizado como “MS-DOS”.

En 1981 se introdujo al mercado el equipo de IBM con MS-DOS, el cual presento un idioma completamente nuevo al público. Escribir "C:" y otros comandos de cifrado el cual se 
convirtieron en parte del diario trabajo. También presentaron al público (\textbackslash) tecla barra diagonal invertida.

Este sistema operativo es eficiente pero muy complejo para muchas personas ya que este era difícil de entender. Microsoft tendría que desarrollar un mejor sistema para las 
personas si quería llegar a cada uno de los hogares del mundo. 

Microsoft empieza a trabajar en la primera versión de un nuevo sistema operativo su nombre de código es Interface manager considerado como su nombre final, pero aparece al 
que hoy todos conocemos como Windows por que describe mejor sus ventanas informáticas este sistemas operativo es anunciado en el año 1983 pero su desarrollo es muy lento el 
cual muchos lo llaman "Vaporware".

Dos años después de su anuncio más exactamente el 20 de Noviembre de 1985 Microsoft lanza Windows 1.0. Un sistema operativo totalmente renovado y mucho más accesible para 
las personas en vez de tener que utilizar comandos de MS-DOS, simplemente se  necesita mover un mouse para apuntar y hacer doble clic para abrir o seleccionar las ventanas 
que deseen. Bill Gates señala "Es un Software único, diseñado para el usuario de equipos serio".

Algunas de las novedades de este sistema operativo es que posee menús desplegables, barras deslizantes, iconos y cuadros de dialogo que hacen que los programas sean más 
fáciles de aprender y usar. Se puede cambiar entre varios programas sin tener que cerrar y volver a iniciar cada uno, Windows 1.0 fue presentado con varios programas entre 
los que se destacan el administrador de archivos MS-DOS, paint, windows writer, notepad, calculadora calendario, un reloj el cual fue introducido para que los usuarios 
administraran sus actividades diarias y tiene además un juego  llamado Reversi. Este Sistema operativo tiene mas facilidades el cual empieza a cumplir con sus objetivos que 
es llegar a tener una computadora en cada uno de los hogares ya que es práctico y fácil de usar para el usuario pero esto es solo el comienzo de windows el cual compartirá 
con nosotros en el diario trabajo facilitándonos grandemente nuestra forma de vida.




