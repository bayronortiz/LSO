\chapter*{Linux y la Educación}
GNU/Linux es un sistema operativo de uso libre con una gran
variedad de funcionalidades que permite a los usuarios estar en
constante interacción con las aplicaciones que brinda, de
manera que puedan tanto visualizar como modificar los códigos
fuentes implementados por cada programador. Esto supone en
contrariedad con el software privado, una mayor fuente de
conocimiento que no inhibe ni limita el anhelo por leer codigo.

No se trata de sustituir un sistema operativo por otro porque ,
tal vez, sea más barato, seguro y fiable, sino de transmitir el
espíritu de colaboración y cooperatividad que es base de toda
empresa de conocimiento. El software libre es inherente a la
educacion por los valores que le guardan.

El software libre permite a los usuarios la libertad de controlar
sus ordenadores, y cooperar unos con otros sin tener
restricciones de ninguna índole. Además que supone un ahorro
significativo a la hora de copiar y redistribuir el software dentro
del sistema educativo.

Pero lo mencionado anteriormente es secundario al verdadero
objetivo por el que linux deber ser considerado un medio virtual
de aprendizaje significativo en constante evolución, esto es, la
implementación del uso continuo de software libre, no como un
sistema operativo que mejora la educación, sino que reemplaza
una educación limitada por una global y extendida.


\section*{Algunos Programas de Contexto Educativo Implementados en GNU/LINUX}
