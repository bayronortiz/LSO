\chapter*{LINUX Y LA EDUCACIÓN}
GNU/Linux es un sistema operativo de uso libre con una gran
variedad de funcionalidades que permite a los usuarios estar en
constante interacción con las aplicaciones que brinda, de
manera que puedan tanto visualizar como modificar los códigos
fuentes implementados por cada programador. Esto supone en
contrariedad con el software privado, una mayor fuente de
conocimiento que no inhibe ni limita el anhelo por leer codigo.

No se trata de sustituir un sistema operativo por otro porque ,
tal vez, sea más barato, seguro y fiable, sino de transmitir el
espíritu de colaboración y cooperatividad que es base de toda
empresa de conocimiento. El software libre es inherente a la
educacion por los valores que le guardan.

El software libre permite a los usuarios la libertad de controlar
sus ordenadores, y cooperar unos con otros sin tener
restricciones de ninguna índole. Además que supone un ahorro
significativo a la hora de copiar y redistribuir el software dentro
del sistema educativo.

Pero lo mencionado anteriormente es secundario al verdadero
objetivo por el que linux deber ser considerado un medio virtual
de aprendizaje significativo en constante evolución, esto es, la
implementación del uso continuo de software libre, no como un
sistema operativo que mejora la educación, sino que reemplaza
una educación limitada por una global y extendida.

Ahora analicemos cuestiones más profundas que nos permitiran
comprender el error en el que esta sometida nuestra sociedad
educativa, y por tanto, inculcada a los estudiantes.
La mision social de las escuelas es enseñar a ser ciudadanos de
una sociedad fuerte, independiente, capaz y libre; ¿libre?, esta
es la palabra clave en la cual debemos cuestionarnos y
preguntarnos , ¿qué clase de libertad tengo, si me cohiben de
ser empírico a la hora de extender mi conocimiento, visualizando
comportamientos internos de programas de gran importancia en
el mundo virtual?. Por ende el sistema educativo debe promover
el uso de software libre como promueven el voto.

Enseñando el software libre, pueden formar ciudadanos
integrales preparados para vivir en una sociedad digital libre de
yugos, a los cuales, las megacorporaciones nos quieren someter.
Por el contrario, al enseñar el uso del software no libre (privado
y de pago), se inculca la dependencia, lo cual se opone a la
misión de las empresas de conocimiento.

El software libre anima a todos a aprender: este repudia la
“encriptación de la tecnología y el saber” que mantiene a los
usuarios en la ignorancia del funcionamiento de la tecnología.
Por el contrario, el software privado abstiene a los usuarios de
adquirir conocimiento empírico.

El software libre en la educación es un medio interactivo para
enseñar y difundir el conocimiento en las escuelas; emitiendo
valores importantes como la solidaridad, libertad y trabajo
colaborativo; Solidario debido a que el usuario puede hacer
copias y distribuirlos libremente, al igual que modificarlas y
difundir diferentes versiones, cooperando al desarrollo
educacional de la comunidad.

La libertad del usuario es la mayor característica en el software
libre; puedes ejecutar el programa como desees, estudiar el
código fuente, cambiarlo y acomodarlo a tus necesidades y
gustos. Respecto al trabajo colaborativo se ve reflejado en las
miles de aplicaciones y obras libres, disponibles para usar,
copiar y modificar; esto hecho por personas de todo el planeta
con el ánimo de colaborar en la educación, el trabajo
comunitario y colaborativo.

El software libre significa un ahorro económico para los
institutos de enseñanza, además contribuirán al progreso de los
más brillantes en programación; los jóvenes despiertan a una
temprana edad un gran interés por saber todo acerca de la
computación y para ello es necesario leer código, modificar y
aportar nuevas ideas; y el software libre ofrece esta
oportunidad.
Pero sin duda la razón más profunda para utilizar software libre
en las escuelas es la educación moral, es enseñar a ser
ciudadanos; en el entorno informático esto se traduce en
instruir a compartir el software. 
Citando las palabras de Richard Stallman, “si traéis software a la escuela, debéis
compartirlo con los demás compañeros, y debéis mostrar el
código fuente en clase, por si alguien quiere aprender. Por lo
tanto, no está permitido traer a la escuela software que no sea
libre a menos que sirva para hacer un trabajo de ingeniería
inversa”.

Ocultar el conocimiento nunca ha formado parte de los
manuales ni la ética profesional, es la búsqueda dinámica y
transparente del conocimiento lo que se comparte por la
comunidad, pues es su mayor activo económico y cultural.
Existen grupos de usuarios muy activos y organizados que se
ayudan entre sí. Si uno tiene un problema puede dirigirse a ellos
para tratar de resolverlo.
Es un sistema seguro y fiable, el alumno no puede dañar el
sistema ni voluntaria, ni accidentalmente. Los niveles de
seguridad son muy altos tales que no será necesario reinstalar el
software.

\section*{ALGUNOS PROGRAMAS EDUCATIVOS IMPLEMENTADOS EN GNU/LINUX}
\subsection*{DEBIAN-JUNIOR}
Personalización de Debian para niños.
Consiste en ayudar a los niños en su proceso de familiarización
con el sistema operativo de manera que puedan adquirir algunas
habilidades y experiencias que tenemos como adultos, ademas
de transmitirles algunos valores como el amor por la libertad y
el fuerte sentido de comunidad.
Algunos beneficios y características de esta distribucion de linux
son:
1. Elegir un tema de escritorio y fondo.
2. Creación personalizada para niños pequeños de entornos de
escritorio.
\subsection*{EDUBUNTU}
Distribucion basada en Ubuntu.
Es una derivación oficial de la distribucion Linux Ubuntu
destinada para ambientes escolares, desarrollada por docentes y
tecnólogos de diferentes países, esto, con el objeto de tener
varias perspectivas de las metodologías pedagógicas que se
utilizan para el aprendizaje de población de entre 6 y 18 años.
La ultima version de esta distribucion es Edubuntu 13.04.

Los objetivos principales de Edubuntu son crear una
centralización administrativa de todos los procesos, usuarios y
demás configuraciones, de un laboratorio de cómputo, con el fin
de poder trabajar en un ambiente de colaboración en clase.
Ademas de coleccionar el mejor software libre con fines
educativos.

\subsection*{EDULINUX}
Distribucion basada en Ubuntu creada por el
Ministerio de Educación de Chile y el Instituto de Informática
Educativa.
Esta distribucion surgió por la necesidad de reutilizar
computadoras antiguas instaladas en la Red Escolar Enlaces de
Chile.

Se materializo a traves una Distribucion Educativa del Sistema
Operativo Linux, que contiene un grupo de aplicaciones libres,
para internet, software de ofimática, un paquete de software
educativo que proviene del proyecto KDE Edutaiment, entre
otros paquetes. Esta distribucion consiste en un sistema
cliente /servidor que funciona a traves de un computador
potente (servidor); potente en el sentido de que posee un
procesador multinucleos, memoria aleatoria y disco duro de
gran capacidad, etc. Y diversos computadores antiguos(clientes),
que conectados todos al servidor, se genera un mayor
aprovechamiento, que por si solos no pueden utilizar aplicacion
modernas.

\subsubsection*{ALGUNAS CONTROVERSIAS POR SU NOMBRE}

Se suele confundir esta distribucion con otra, de origen
canadiense, antiguamente denominada EduLinux, pero que
actualmente se denomina LinuxEdu-Québec.
Por otro lado, existe otra distribucion de origen polaco
denominada LinuxEdu-CD.


